% Set up the document's format to A4 and the font's size to 12pt.
\documentclass[a4paper,12pt]{report}

% Set up the input's encoding to UTF-8, the document's font and language to T1 (adapted to french) and french (the grammar linter uses this parameter).
\usepackage[utf8]{inputenc}
\usepackage[T1]{fontenc}
\usepackage[british,french]{babel}

\usepackage[dvipsnames]{xcolor}

% Set up the document's margins.
\usepackage{geometry}
\geometry{hmargin=1.5cm,vmargin=1.5cm}

% Set up the document's title, author and date.
\title{Physique -- MPI}
\author{Romain Bricout}
\date{\today}

% The three main maths packages. They are used for a lot of things.
\usepackage{amssymb,amsmath}
\usepackage{mathtools}

% Useful to create nice and easy signs or variations tables.
\usepackage{tkz-tab}

% Useful to create any kind of visual representation (graph functions, illustrate geometry problems, etc)
\usepackage{tikz}
\usepackage{tikz-3dplot}
\usepackage{pgfplots}
\pgfplotsset{compat=newest}
\usepgfplotslibrary{fillbetween}
\usetikzlibrary{patterns,patterns.meta,angles,quotes,arrows,arrows.meta,bending,decorations.markings,babel,decorations.pathreplacing,calligraphy,shapes.misc}
\tikzset{cross/.style={cross out, draw=black, minimum size=2*(#1-\pgflinewidth), inner sep=0pt, outer sep=0pt},
%default radius will be 1pt.
cross/.default={1pt}}
\tikzstyle{axe}=[->,thick,black]
\tdplotsetmaincoords{60}{110}

% Allows to edit the itemize environment's default item document-wide.
\usepackage{enumitem}

% Allows to define \notfoo or \nfoo (not recommended) in order for \not\foo to work as wished.
\usepackage{newtxmath}

\DeclareSymbolFont{CMletters}{OML}{cmm}{m}{it}
\DeclareMathSymbol{\nu}{\mathord}{CMletters}{23}
\DeclareMathSymbol{\delta}{\mathord}{CMletters}{14}
\DeclareMathSymbol{\zeta}{\mathord}{CMletters}{16}
\DeclareMathSymbol{\kappa}{\mathord}{CMletters}{20}
\DeclareMathSymbol{\xi}{\mathord}{CMletters}{24}
\DeclareMathSymbol{\pi}{\mathord}{CMletters}{25}
\DeclareMathSymbol{\upsilon}{\mathord}{CMletters}{29}
\DeclareMathSymbol{\chi}{\mathord}{CMletters}{31}
\DeclareMathSymbol{\omega}{\mathord}{CMletters}{33}
\DeclareMathSymbol{\Phi}{\mathord}{CMletters}{8}
\DeclareMathSymbol{\mOmega}{\mathord}{CMletters}{10}
\DeclareMathSymbol{\lambda}{\mathord}{CMletters}{21}

% Makes the table of contents clickable and gives useful commands for links in general.
\usepackage{hyperref}
\hypersetup{colorlinks=false,linktoc=all}

% Gives the llbracket and rrbracket commands for integer intervals.
\usepackage{stmaryrd}

% Useful to insert nice-looking quotes.
\usepackage{epigraph}

% Allows to insert chapter-specific table of contents.
\usepackage{minitoc}
\mtcselectlanguage{french}
\setcounter{minitocdepth}{6}

% Useful when units are needed.
\usepackage{siunitx}
\sisetup{
locale=FR,
detect-all,
inter-unit-product=\ensuremath{\cdot},
list-final-separator={et},
list-pair-separator={et},
range-phrase={\ensuremath{\xleftrightarrow{}}},
exponent-product=\ensuremath{\cdot},
per-mode=power-positive-first,
input-digits = 0123456789\pi
}

\usepackage[thmmarks]{ntheorem}
\makeatletter
\let\old@thm\@thm
\usepackage[lowercase]{theoremref}
\def\@thm#1#2#3{\def\thmref@currname{#3}\old@thm{#1}{#2}{#3}}
\makeatother

% Allows whiteboard digits with \mathds
\usepackage{dsfont}

\usepackage{needspace}

% Useful for better-looking oneline fractions
\usepackage{nicefrac}

% Set up the horizontal space before the first line of a new paragraph to 2em and the vertical space between two paragraphs to 1em.
\setlength{\parindent}{0pt}
\setlength{\parskip}{1em}

% Adds 0.5em to the vertical space between two lines in an align environment. It looks better.
\addtolength{\jot}{0.5em}

% Allows align environment to break if it's too long to fit in the page where it began.
\allowdisplaybreaks[1]

% Trick to make semicolons considered like relation operators (such as =) and therefore being equidistantly spaced from the two numbers around it.
\mathcode`;=\numexpr\mathcode`;-"3000

% Commands for size-adaptative parentheses, brackets, curly brackets, absolute value and magnitude.
\newcommand{\paren}[1]{\left(#1\right)} % (x)
\newcommand{\croch}[1]{\left[#1\right]} % [x]
\newcommand{\accol}[1]{\left\lbrace#1\right\rbrace} % {x}
\newcommand{\abs}[1]{\left\lvert#1\right\rvert} % |x|
\newcommand{\norme}[1]{\left\|#1\right\|} % ||x||
\newcommand{\floor}[1]{\left\lfloor#1\right\rfloor} % ⌊x⌋
\newcommand{\ceil}[1]{\left\lceil#1\right\rceil} % ⌈x⌉

% Commands for size-adaptative intervals and integer intervals. The commands' roots are "interv" and "interventier" and the added e or i at the end mean "excluded" and "included" respectively.
\newcommand{\intervii}[2]{\left[#1;#2\right]} % [a;b]
\newcommand{\intervee}[2]{\left]#1;#2\right[} % ]a;b[
\newcommand{\intervie}[2]{\left[#1;#2\right[} % [a;b[
\newcommand{\intervei}[2]{\left]#1;#2\right]} % ]a;b]
\newcommand{\interventierii}[2]{\left\llbracket#1;#2\right\rrbracket} % non-ASCII characters needed
\newcommand{\interventieree}[2]{\left\rrbracket#1;#2\right\llbracket} % non-ASCII characters needed
\newcommand{\interventierie}[2]{\left\llbracket#1;#2\right\llbracket} % non-ASCII characters needed
\newcommand{\interventierei}[2]{\left\rrbracket#1;#2\right\rrbracket} % non-ASCII characters needed

% Commands for usually used sets.
\newcommand{\N}{\mathbb{N}} % natural integers
\newcommand{\Ns}{\mathbb{N}^*}

\newcommand{\Z}{\mathbb{Z}} % relative integers
\newcommand{\Zp}{\mathbb{Z}_+}
\newcommand{\Zs}{\mathbb{Z}^*}
\newcommand{\Zps}{\mathbb{Z}_+^*}

\newcommand{\D}{\mathbb{D}} % decimal numbers
\newcommand{\Dp}{\mathbb{D}_+}
\newcommand{\Dm}{\mathbb{D}_-}
\newcommand{\Ds}{\mathbb{D}^*}
\newcommand{\Dps}{\mathbb{D}_+^*}
\newcommand{\Dms}{\mathbb{D}_-^*}

\newcommand{\Q}{\mathbb{Q}} % rational numbers
\newcommand{\Qp}{\mathbb{Q}_+}
\newcommand{\Qm}{\mathbb{Q}_-}
\newcommand{\Qs}{\mathbb{Q}^*}
\newcommand{\Qps}{\mathbb{Q}_+^*}
\newcommand{\Qms}{\mathbb{Q}_-^*}

\newcommand{\R}{\mathbb{R}} % real numbers
\newcommand{\Rp}{\mathbb{R}_+}
\newcommand{\Rm}{\mathbb{R}_-}
\newcommand{\Rs}{\mathbb{R}^*}
\newcommand{\Rps}{\mathbb{R}_+^*}
\newcommand{\Rms}{\mathbb{R}_-^*}
\newcommand{\Rb}{\overline{\mathbb{R}}}

\newcommand{\C}{\mathbb{C}} % complex numbers
\newcommand{\Cs}{\mathbb{C}^*}

\newcommand{\U}{\mathbb{U}} % complex numbers whose modulus is 1
\renewcommand{\H}{\mathbb{H}} % quaternions (\H normally prints slanted quotation marks)
\renewcommand{\O}{\mathbb{O}} % octonions (\O normally prints a slashed capital o : Ø)
\newcommand{\M}{\mathcal{M}} % matrices
\newcommand{\GL}{\mathrm{GL}} % invertible matrices
\renewcommand{\S}{\mathcal{S}} % solutions of an equation (\S normally prints a silcrow : §)

\renewcommand{\P}[1]{\mathcal{P}\paren{#1}} % subsets of a set
\newcommand{\F}[2]{\mathcal{F}\paren{#1,#2}} % functions from 1 to 2
\newcommand{\V}[1]{\mathcal{V}\paren{#1}} % neighborhood of a number

% Redefines \Re and \Im to print Re and Im (the same way as ln or lim) instead of fraktur R and I which don't look nice and are less readable.
\renewcommand{\Re}{\operatorname{Re}}
\renewcommand{\Im}{\operatorname{Im}}

% Command to print an upright e for the exponential instead of a slanted e and put the exponent.
\newcommand{\e}[1]{\mathrm{e}^{#1}}

% Command to print the imaginary i with a little space on the right. This way, the exponents don't look confusing. \i normally prints a dotless i.
\renewcommand{\i}{i\mkern1mu}

% Commands for 2D and 3D vectors' coordinates
\newcommand{\dcoords}[2]{\begin{pmatrix}#1\\#2\end{pmatrix}}
\newcommand{\tcoords}[3]{\begin{pmatrix}#1\\#2\\#3\end{pmatrix}}

% Redefines binom to print nicer parentheses around the numbers.
\renewcommand{\binom}[2]{\begin{pmatrix}#2\\#1\end{pmatrix}}

% Command for a QED black square. It automatically prints a whitespace before the square such that it looks nice.
\newcommand{\cqfd}{\text{ }\blacksquare}

% Commands with more explicit names for the best way to express divisibility (mid and nmid).
\newcommand{\divise}{\mid}
\newcommand{\notdivise}{\nmid}

% Commands that do the exact same thing but with explicit names for a complex number's conjugate and an event's negation in probability.
\newcommand{\conj}[1]{\overline{#1}}

% Command for a size-adaptative middle bar meaning "such that" (with spacing around it in order to look nice).
\newcommand{\tq}{\;\middle|\;}

% Command with an explicit name for the scalar product.
\newcommand{\scal}{\cdot}
\newcommand{\vecto}{\operatorname{_\wedge}}

% Shortcut for forcing displaystyle in inline mode.
\newcommand{\ds}{\displaystyle}

% Make the not version of implies, impliedby and iff look nice.
\newcommand{\notimplies}{\centernot{\imp}}
\newcommand{\notimpliedby}{\centernot{\impr}}
\newcommand{\notiff}{\centernot{\ssi}}

% Shortcut for P(event).
\newcommand{\proba}[1]{P\paren{#1}}

% More explicit names for land (logical and) and lor (logical or).
\newcommand{\et}{\land}
\newcommand{\ou}{\lor}
\newcommand{\non}{\lnot}

% Explicitly named environment for tkz-tab tables. Automatically centers the table and handles the tikzpicture environment.
\newenvironment{tkz}
{
\begin{center}
\begin{tikzpicture}
}
{
\end{tikzpicture}
\end{center}
}

% More explicitly named commands for the creation of tkz-tab tables.
\newcommand{\tableauinit}[2]{\tkzTabInit{#1}{#2}}
\newcommand{\tableausignes}[1]{\tkzTabLine{#1}}
\newcommand{\tableauvariations}[1]{\tkzTabVar{#1}}

% Shortcut for the curve and the domain of the given function.
\newcommand{\graphe}[1]{\Gamma_{#1}}
\newcommand{\ensembledef}[1]{\mathcal{D}_{#1}}

% Various environments that create boxes. Each one is one type of thing (example, proof, etc). Each type has its own automatic counter.
\theoremstyle{break}
\theorembodyfont{\upshape}
\theoremheaderfont{\itshape}
\theoremprework{\bigskip\needspace{\baselineskip}\color{green}\hrule\color{black}}
\theorempostwork{\bigskip}
\newtheorem{rem}{Remarque}[chapter]

\theoremstyle{break}
\theorembodyfont{\upshape}
\theoremheaderfont{\itshape}
\theoremprework{\bigskip\needspace{\baselineskip}\color{green}\hrule\color{black}}
\theorempostwork{\bigskip}
\newtheorem{ex}{Exemple}[chapter]

\theoremstyle{break}
\theorembodyfont{\upshape}
\theoremheaderfont{\itshape}
\theoremprework{\bigskip\needspace{\baselineskip}\color{green}\hrule\color{black}}
\theorempostwork{\bigskip}
\newtheorem{rappel}{Rappel}[chapter]

\theoremstyle{break}
\theorembodyfont{\upshape}
\theoremheaderfont{\itshape}
\theoremprework{\bigskip\needspace{\baselineskip}\color{brown}\hrule\color{black}}
\theorempostwork{\bigskip}
\newtheorem{oubli}{Oubli}[chapter]

\theoremstyle{break}
\theorembodyfont{\upshape}
\theoremheaderfont{\normalfont\bfseries}
\theoremprework{\bigskip\needspace{\baselineskip}\color{blue}\hrule\color{black}}
\theorempostwork{\bigskip}
\newtheorem{defi}{Définition}[chapter]

\theoremstyle{break}
\theorembodyfont{\upshape}
\theoremheaderfont{\normalfont\bfseries}
\theoremprework{\bigskip\needspace{\baselineskip}\color{blue}\hrule\color{black}}
\theorempostwork{\bigskip}
\newtheorem{defprop}{Définition/Proposition}[chapter]

\theoremstyle{break}
\theorembodyfont{\upshape}
\theoremheaderfont{\normalfont\bfseries}
\theoremprework{\bigskip\needspace{\baselineskip}\color{blue}\hrule\color{black}}
\theorempostwork{\bigskip}
\newtheorem{nota}{Notation}[chapter]

\theoremstyle{break}
\theorembodyfont{\itshape}
\theoremheaderfont{\normalfont\bfseries}
\theoremprework{\bigskip\needspace{\baselineskip}\color{red}\hrule\color{black}}
\theorempostwork{\bigskip}
\newtheorem{theo}{Théorème}[chapter]

\theoremstyle{break}
\theorembodyfont{\itshape}
\theoremheaderfont{\normalfont\bfseries}
\theoremprework{\bigskip\needspace{\baselineskip}\color{red}\hrule\color{black}}
\theorempostwork{\bigskip}
\newtheorem{prop}{Proposition}[chapter]

\theoremstyle{break}
\theorembodyfont{\itshape}
\theoremheaderfont{\normalfont\bfseries}
\theoremprework{\bigskip\needspace{\baselineskip}\color{red}\hrule\color{black}}
\theorempostwork{\bigskip}
\newtheorem{cor}{Corollaire}[chapter]

\theoremstyle{break}
\theorembodyfont{\upshape}
\theoremheaderfont{\normalfont\bfseries}
\theoremprework{\bigskip\needspace{\baselineskip}\color{purple}\hrule\color{black}}
\theorempostwork{\bigskip}
\newtheorem{meth}{Méthode}[chapter]

\theoremstyle{nonumberbreak}
\theorembodyfont{\upshape}
\theoremheaderfont{\itshape}
\theoremsymbol{\ensuremath{\cqfd}}
\theoremprework{\bigskip\needspace{\baselineskip}\color{yellow}\hrule\color{black}}
\theorempostwork{\bigskip}
\newtheorem{dem}{Démonstration}[chapter]

% Commands to make proofs easier to write
\newcommand{\impdir}{\fbox{\(\imp\)}~}
\newcommand{\imprec}{\fbox{\(\impr\)}~}
\newcommand{\incdir}{\fbox{\(\subset\)}~}
\newcommand{\increc}{\fbox{\(\supset\)}~}
\newcommand{\unicite}{\fbox{unicité}~}
\newcommand{\existence}{\fbox{existence}~}

\renewcommand{\to}{\longrightarrow}
\renewcommand{\mapsto}{\longmapsto}

\newcommand{\fonction}[5]{\begin{array}[t]{cccl}#1 : & #2 & \to & #3 \\ & #4 & \mapsto & #5\end{array}}
\newcommand{\fonctionlambda}[4]{\begin{array}[t]{ccl}#1 & \to & #2 \\ #3 & \mapsto & #4\end{array}}

\renewcommand{\subset}{\subseteq}
\renewcommand{\supset}{\supseteq}

\renewcommand{\leq}{\leqslant}
\renewcommand{\geq}{\geqslant}

\newcommand{\pinf}{+\infty}
\newcommand{\minf}{-\infty}

\newcommand{\id}[1]{\mathrm{id}_{#1}}

\renewcommand{\phi}{\varphi}
\renewcommand{\epsilon}{\varepsilon}

\newcommand{\ind}[1]{\mathds{1}_{#1}}

\newcommand{\iR}{\i\R}

\newcommand{\tcheby}[2]{T_{#1}\paren{#2}}
\newcommand{\utcheby}[2]{U_{#1}\paren{#2}}

\mathcode`l="8000
\begingroup
\makeatletter
\lccode`\~=`\l
\DeclareMathSymbol{\lsb@l}{\mathalpha}{letters}{`l}
\lowercase{\gdef~{\ifnum\the\mathgroup=\m@ne \ell \else \lsb@l \fi}}%
\endgroup

\newcommand{\ensvide}{\emptyset}

\newcommand{\rond}{\circ}

\newcommand{\union}{\cup}
\newcommand{\inter}{\cap}
\newcommand{\bigunion}{\bigcup}
\newcommand{\biginter}{\bigcap}

\newcommand{\ssi}{\iff}
\newcommand{\imp}{\implies}
\newcommand{\impr}{\impliedby}

\newcommand{\excluant}{\setminus}

\newcommand{\littletaller}{\mathchoice{\vphantom{\big|}}{}{}{}}
\newcommand{\restr}[2]{{
\left.\kern-\nulldelimiterspace#1\littletaller\right|_{#2}
}}
\newcommand{\corestr}[2]{{
\left.\kern-\nulldelimiterspace#1\littletaller\right|^{#2}
}}
\newcommand{\restrbar}[1]{{
\left.\kern-\nulldelimiterspace#1\littletaller\right|
}}

\newcommand{\rel}{\mathcal{R}}

\newcommand{\classesdequiv}[1]{\nicefrac{#1}{\sim}}

\newcommand{\majo}[1]{\mathrm{majorants}\paren{#1}}
\newcommand{\mino}[1]{\mathrm{minorants}\paren{#1}}

\newcommand{\ensdiv}[1]{\operatorname{div}#1}

\newcommand{\E}[1]{\times 10^{#1}}

\newcommand{\siecle}[1]{\textsc{#1}\ieme~}

\usepackage{derivative}
\derivset{\pdv}[delims-eval=.)]
\derivset{\odv}[delims-eval=.)]

\newcommand{\moy}[1]{\left\langle#1\right\rangle}

\newcommand{\cte}{\mathrm{cte}}
\newcommand{\ext}{\mathrm{ext}}
\newcommand{\inte}{\mathrm{int}}
\newcommand{\ncons}{\mathrm{nc}}
\newcommand{\cons}{\mathrm{c}}
\newcommand{\eq}{\mathrm{eq}}
\newcommand{\ther}{\mathrm{th}}
\newcommand{\univ}{\mathrm{univ}}
\newcommand{\maxi}{\mathrm{max}}
\newcommand{\mini}{\mathrm{min}}
\newcommand{\geo}{\mathrm{geo}}
\newcommand{\grav}{\mathrm{grav}}
\newcommand{\pole}{\mathrm{pole}}
\newcommand{\equateur}{\mathrm{equateur}}
\newcommand{\ie}{\textit{i.e.} }
\newcommand{\cf}{\textit{cf.} }
\newcommand{\sortant}{\mathrm{sortant}}
\newcommand{\entrant}{\mathrm{entrant}}
\newcommand{\enl}{\mathrm{enl}}
\newcommand{\limite}{\mathrm{lim}}
\newcommand{\tot}{\mathrm{tot}}
\newcommand{\sortie}{\mathrm{sortie}}
\newcommand{\cedee}{\mathrm{cedee}}

\usepackage{diagbox}

\usepackage{witharrows}

\newcommand{\dif}{\mathrm{d}}

\DeclareSIUnit{\cal}{\text{cal}}
\DeclareSIUnit{\Longueur}{L}
\DeclareSIUnit{\Temps}{T}
\DeclareSIUnit{\Masse}{M}
\DeclareSIUnit{\Temperature}{\Theta}
\DeclareSIUnit{\IntensiteElec}{I}
\DeclareSIUnit{\Quantite}{N}
\DeclareSIUnit{\IntensiteLumi}{J}

\newcommand{\guillemets}[1]{\og #1 \fg{}}

\usepackage{abstract}
\addto\captionsfrench{\renewcommand{\abstractname}{\Large Introduction}}

\hypersetup{
pdftitle=Physique MPI,
pdfauthor=Romain Bricout
}

\newenvironment{Tabular}[2][1]{\def\arraystretch{#1}\tabular{#2}}{\endtabular}

\makeatletter
\@addtoreset{chapter}{part}
\makeatother

\newcommand{\supinf}{\lessgtr}

\usepackage[europeanresistors,americaninductors,straightvoltages,siunitx,RPvoltages]{circuitikz}

\newenvironment{circuit}{\begin{center}\begin{circuitikz}}{\end{circuitikz}\end{center}}

\usepackage{bclogo}

\newcommand{\attention}{\bcattention\;\;}

\setcounter{secnumdepth}{3}

\newcommand{\note}[1]{\textbf{\(\star\star\) #1 \(\star\star\)}}
\newcommand{\cad}{c'est-à-dire }
\newcommand{\Cad}{C'est-à-dire }

\usepackage{titletoc}
\dottedcontents{section}[5.5em]{}{3.2em}{1pc}

\usepackage{microtype}

\newcommand{\prim}{^{\,\prime}}
\newcommand{\seconde}{^{\,\prime\prime}}

\usepackage{cancel}
\renewcommand{\CancelColor}{\blue}

\ExplSyntaxOn
\RenewDocumentCommand{\v}{m}{
    \int_compare:nTF { \tl_count:n { #1 } > 1 }
    {
        \overrightarrow{#1}
    }
    {
        \vec{#1}
    }
}
\ExplSyntaxOff

\newcommand{\PFD}{\(\v{\mathrm{PFD}}\) }

\DeclareMathOperator{\Arctan}{Arctan}
\DeclareMathOperator{\Arcsin}{Arcsin}
\DeclareMathOperator{\Arccos}{Arccos}
\DeclareMathOperator{\cotan}{cotan}
\DeclareMathOperator{\sh}{sh}
\DeclareMathOperator{\ch}{ch}

\newcommand{\call}[1]{\mathscr{#1}}

\DeclareMathOperator{\rot}{\v{\mathrm{rot}}}
\DeclareMathOperator{\grad}{\v{\mathrm{grad}}}
\let\div\relax % workaround to redeclare
\DeclareMathOperator{\div}{\mathrm{div}}
\DeclareMathOperator{\lap}{\Delta}
\DeclareMathOperator{\lapv}{\v{\operatorname{\Delta}}}
\DeclareMathOperator{\dalemb}{\Box}

\newcommand{\MG}{(MG) }
\newcommand{\MT}{(MT) }
\newcommand{\MA}{(\(\v{\mathrm{MA}}\)) }
\newcommand{\MF}{(\(\v{\mathrm{MF}}\)) }

\usepackage{makecell}

\begin{document}
\renewcommand{\labelitemi}{\(\triangleright\)}
\renewcommand{\labelenumi}{(\arabic{enumi})}

\everymath{\ds}

\maketitle

\begin{abstract}
Ce document réunit l'ensemble de mes cours de Physique de MPI. Le professeur était M. Jacob. J'ai adapté certaines formulations me paraissant floues ou ne me plaisant pas mais le contenu pur des cours est strictement équivalent. Le document est organisé selon la hiérarchie suivante : partie, chapitre, I), 1), a). Les parties sont celles du programme : mécanique, optique, ... Leur ordre dans ce document est arbitraire et ne reflète pas l'ordre de traitement des chapitres durant l'année.

Les éléments des tables des matières initiale et présentes au début de chaque chapitre sont cliquables (amenant directement à la partie cliquée).

Code couleur des vecteurs (et flèches en général) : \begin{itemize}
\item noir : légendes ou mouvements ;
\item violet : champs de pesanteur ;
\item rouge : forces, vecteurs densité de courant volumique ;
\item bleu : vecteurs unitaires, champs électriques ;
\item vert : vecteurs vitesse ou accélération, champs magnétiques ;
\item jaune : vecteurs rotation. \\ % Goldenrod
\end{itemize}
\end{abstract}

\dominitoc\tableofcontents

\part{Mécanique}

\chapter{Référentiels non-galiléens}

\minitoc

\section*{Introduction}
\addcontentsline{toc}{section}{Introduction}

Rappel :

Un référentiel est un système de trois axes de coordonnées lié à un solide de référence (l'observateur) et munie d'une horloge mesurant le temps.

Un référentiel est dit galiléen (ou inertiel) si le principe d'inertie y est vérifié.

Tout référentiel en translation rectiligne uniforme par rapport à un référentiel galiléen est galiléen. Le principe fondamental de la dynamique et le principe d'inertie y sont vérifiés.

Cependant, il est parfois judicieux d'analyser le mouvement d'un solide qui a un mouvement quelconque dans un référentiel galiléen dans un autre référentiel où le mouvement se décrit simplement.

Par exemple, dans le référentiel terrestre, la valve d'une roue de vélo qui tourne a une trajectoire cycloïdale alors que dans le référentiel de la roue, elle a une trajectoire circulaire.

\section{Changement de référentiel}

\subsection{Loi de composition des vitesses}

\subsubsection{Cas général}

Soient deux référentiels \(R\) et \(R\prim\) associés à deux repères \(\paren{Ox,Oy,Oz}\) et \(\paren{O\prim x\prim,O\prim y\prim,O\prim z\prim}\) et munis d'horloges identiques \(H\) et \(H\prim\).

Dans le cadre de la cinématique classique (\ie non-relativiste), \(H=H\prim\) donc \(t=t\prim\) : le temps est absolu.

C'est valable si \(v\ll c\) (en pratique, \(v<\num{0.1}c\)).

\begin{center}
\begin{tikzpicture}[scale=2,tdplot_main_coords]
\coordinate (O) at (0,0,0) node[left] {\(O\)};
\draw[axe] (O) -- ++(2,0,0) node[anchor=north east] {\(x\)};
\draw[axe] (O) -- ++(0,2,0) node[anchor=north west] {\(y\)};
\draw[axe] (O) -- ++(0,0,2) node[anchor=south] {\(z\)};

\coordinate (O') at (0,5,0);
\draw[axe] (O') -- ++(2,0,0) node[anchor=north east] {\(x\prim\)};
\draw[axe] (O') -- ++(0,2,0) node[anchor=north west] {\(y\prim\)};
\draw[axe] (O') -- ++(0,0,2) node[anchor=south] {\(z\prim\)};
\node[left] at (O') {\(O\prim\)};

\coordinate (M) at (1,6.5,1.5);
\filldraw (M) circle (1pt);
\node[above right] at (M) {\(M\)};

\draw (O) -- (M) -- (O');
\end{tikzpicture}
\end{center}

On a \(\v{OM}=\v{OO\prim}+\v{O\prim M}\).

On dérive dans \(R\) :

On a \[\v{OM}=\v{OO\prim}+x\prim\v{u}_{x\prim}+y\prim\v{u}_{y\prim}+z\prim\v{u}_{z\prim}\] avec \(\paren{\v{u}_{x\prim},\v{u}_{y\prim},\v{u}_{z\prim}}\) une base orthonormale directe de \(R\prim\).

D'où \[\begin{aligned}
\odv{\v{OM}}{t}_R=\odv{\v{OO\prim}}{t}_R&+\odv{x\prim}{t}\v{u}_{x\prim}+\odv{y\prim}{t}\v{u}_{y\prim}+\odv{z\prim}{t}\v{u}_{z\prim} \\
&+x\prim\odv{\v{u}_{x\prim}}{t}_R+y\prim\odv{\v{u}_{y\prim}}{t}_R+z\prim\odv{\v{u}_{z\prim}}{t}_R
\end{aligned}\]

On obtient donc la loi de composition des vitesses : \[\textcolor{red}{\v{v}\paren{M}_R=\v{v}\paren{M}_{R\prim}+\v{v}_e}\] où \begin{description}
    \item[] \(\v{v}_e=\underbrace{\odv{\v{OO\prim}}{t}}_{\substack{\text{déplacement} \\ \text{de }R\prim \\ \text{par rapport } \\ \text{à }R}}+\underbrace{x\prim\odv{\v{u}_{x\prim}}{t}_R+y\prim\odv{\v{u}_{y\prim}}{t}_R+z\prim\odv{\v{u}_{z\prim}}{t}_R}_{\text{rotation de }R\prim\text{ par rapport à }R}\) est la vitesse d'entraînement.
\end{description}

\subsubsection{Cas de \(R\prim\) en translation par rapport à \(R\)}

Si \(R\prim\) est en translation par rapport à \(R\) alors ils partagent la même base donc \(\paren{\v{u}_x,\v{u}_y,\v{u}_z}=\paren{\v{u}_{x\prim},\v{u}_{y\prim},\v{u}_{z\prim}}\).

Donc \[\v{v}_e=\odv{\v{OO\prim}}{t}\]

La vitesse d'entraînement est indépendante du point étudié (tous les points de \(R\prim\) sont entraînés à la même vitesse).

\subsubsection{Cas de \(R\prim\) en rotation uniforme autour d'un axe fixe de \(R\)}

\begin{center}
\begin{tikzpicture}[scale=2,tdplot_main_coords]
\coordinate (O) at (0,0,0);
\draw[axe] (O) -- ++(2,0,0) node[anchor=north east] {\(x\)};
\draw[axe] (O) -- ++(0,2,0) node[anchor=north west] {\(y\)};
\draw[axe] (O) -- ++(0,0,2) node[anchor=south east] {\(z\)};
\node[left] at (O) {\(O\)};
\node[above right, blue] at (O) {\(O\prim\)};
\tdplotdrawarc[->]{(O)}{1.5}{0}{20}{anchor=north}{\(\theta\)};
\tdplotsetrotatedcoords{20}{0}{0};
\draw[tdplot_rotated_coords,axe,color=blue] (O) -- ++(2,0,0) node[anchor=north] {\(x\prim\)};
\draw[tdplot_rotated_coords,axe,color=blue] (O) -- ++(0,2,0) node[anchor=south west] {\(y\prim\)};
\draw[tdplot_rotated_coords,axe,color=blue] (O) -- ++(0,0,2) node[anchor=south west] {\(z\prim\)};
\end{tikzpicture}
\end{center}

On pose \(\omega=\dot{\theta}=\cte\) (rotation uniforme).

On a \(\odv{\v{OO\prim}}{t}=\v{0}\) et comme \(\v{u}_{z\prim}=\v{u}_z\) on a \(\odv{\v{u}_{z\prim}}{t}_R=\v{0}\).

Déterminons \(\odv{\v{u}_{x\prim}}{t}_R\).

On a \(\odv{\v{u}_{x\prim}}{t}=\underbrace{\odv{\theta}{t}}_{=\dot{\theta}=\omega}\odv{\v{u}_{x\prim}}{\theta}_R\).

En projetant dans la base \(\paren{\v{u}_x,\v{u}_y}\), on obtient \[\begin{dcases}
\v{u}_{x\prim}=\cos\theta\v{u}_x+\sin\theta\v{u}_y \\
\v{u}_{y\prim}=-\sin\theta\v{u}_x+\cos\theta\v{u}_y
\end{dcases}\]

Donc \(\odv{\v{u}_{x\prim}}{t}_R=\omega\v{u}_{y\prim}\).

De même : \(\odv{\v{u}_{y\prim}}{t}_R=-\omega\v{u}_{x\prim}\).

Donc \[\begin{aligned}
\v{v}_e&=x\prim\odv{\v{u}_{x\prim}}{t}_R+y\prim\odv{\v{u}_{y\prim}}{t}_R \\
&=x\prim\omega\v{u}_{y\prim}-y\prim\omega\v{u}_{x\prim}
\end{aligned}\]

On pose \(\v{\omega}\) le vecteur rotation de \(R\prim\) par rapport à \(R\) : \(\v{\omega}=\omega\v{u}_z\)

On a alors \[\odv{\v{u}_{x\prim}}{t}_R=\v{\omega}\vecto\v{u}_{x\prim}\qquad\odv{\v{u}_{y\prim}}{t}_R=\v{\omega}\vecto\v{u}_{y\prim}\]

Donc \[\begin{WithArrows}
\v{v}_e&=x\prim\paren{\v{\omega}\vecto\v{u}_{x\prim}}+y\prim\paren{\v{\omega}\vecto\v{u}_{y\prim}} \Arrow{linéarité de \(\vecto\)} \\
&=\v{\omega}\vecto\paren{x\prim\v{u}_{x\prim}+y\prim\v{u}_{y\prim}} \Arrow{\(\v{\omega}\vecto\v{u}_z=\v{0}\)} \\
&=\v{\omega}\vecto\underbrace{\paren{x\prim\v{u}_{x\prim}+y\prim\v{u}_{y\prim}+z\v{u}_z}}_{=\v{OM}}
\end{WithArrows}\]

D'où \[\textcolor{red}{\v{v}_e=\v{\omega}\vecto\v{OM}}\]

\subsection{Loi de composition des accélérations}

\subsubsection{Cas général}

On a \[\begin{WithArrows}
\v{v}\paren{M}_R&=\odv{\v{OO\prim}}{t} \\
&\textcolor{white}{=}+\odv{x\prim}{t}\v{u}_{x\prim}+\odv{y\prim}{t}\v{u}_{y\prim}+\odv{z\prim}{t}\v{u}_{z\prim} \\
&\textcolor{white}{=}+x\prim\odv{\v{u}_{x\prim}}{t}_R+y\prim\odv{\v{u}_{y\prim}}{t}_R+z\prim\odv{\v{u}_{z\prim}}{t}_R \Arrow{\(\odv{}{t}_R\)} \\
\v{a}\paren{M}_R=\odv{\v{v}\paren{M}_R}{t}_R&=\textcolor{blue}{\odv[order=2]{\v{OO\prim}}{t}_R} \\
&\textcolor{white}{=}\textcolor{blue}{+x\prim\odv[order=2]{\v{u}_{x\prim}}{t}_R+y\prim\odv[order=2]{\v{u}_{y\prim}}{t}_R+z\prim\odv[order=2]{\v{u}_{z\prim}}{t}_R} \\
&\textcolor{white}{=}\textcolor{red}{+2\odv{x\prim}{t}\odv{\v{u}_{x\prim}}{t}_R+2\odv{y\prim}{t}\odv{\v{u}_{y\prim}}{t}_R+2\odv{z\prim}{t}\odv{\v{u}_{z\prim}}{t}_R} \\
&\textcolor{white}{=}\textcolor{Green}{+\odv[order=2]{x\prim}{t}\v{u}_{x\prim}+\odv[order=2]{y\prim}{t}\v{u}_{y\prim}+\odv[order=2]{z\prim}{t}\v{u}_{z\prim}}
\end{WithArrows}\]

Donc \[\v{a}\paren{M}_R=\textcolor{blue}{\v{a}_e}+\textcolor{red}{\v{a}_C}+\textcolor{Green}{\v{a}\paren{M}_{R\prim}}\] où \begin{description}
    \item[] \(\v{a}_e\) est l'accélération d'entraînement \\
    \item[] \(\v{a}_C\) est l'accélération de Coriolis \\
    \item[] \(\v{a}\paren{M}_{R\prim}\) est l'accélération relative de \(M\) dans \(R\prim\).
\end{description}

\subsubsection{Cas de \(R\prim\) en translation par rapport à \(R\)}

On a \(\v{u}_{x\prim}=\v{u}_x\), \(\v{u}_{y\prim}=\v{u}_y\) et \(\v{u}_{z\prim}=\v{u}_z\).

Donc \(\odv{\v{u}_{x\prim}}{t}_R=\v{0}\) et \(\odv[order=2]{\v{u}_{x\prim}}{t}_R=\v{0}\). Idem pour \(\v{u}_{y\prim}\) et \(\v{u}_{z\prim}\).

\attention \(O\prim\) n'est pas nécessairement en translation par rapport à \(O\).

On a donc \(\v{a}_C=\v{0}\) et \(\v{a}_e=\odv[order=2]{\v{OO\prim}}{t}_R\) donc \[\v{a}\paren{M}_R=\v{a}\paren{M}_{R\prim}+\odv[order=2]{\v{OO\prim}}{t}_R\]

On remarque que l'accélération d'entraînement ne dépend pas du point de \(R\prim\) étudié.

\subsubsection{Cas de \(R\prim\) en rotation uniforme autour d'un axe fixe de \(R\)}

\begin{center}
\begin{tikzpicture}[scale=2,tdplot_main_coords]
\coordinate (O) at (0,0,0);
\draw[axe] (O) -- ++(2,0,0) node[anchor=north east] {\(x\)};
\draw[axe] (O) -- ++(0,2,0) node[anchor=north west] {\(y\)};
\draw[axe] (O) -- ++(0,0,2) node[anchor=south east] {\(z\)};
\node[left] at (O) {\(O\)};
\node[above right, blue] at (O) {\(O\prim\)};
\tdplotdrawarc[->]{(O)}{1.5}{0}{20}{anchor=north}{\(\theta\)};
\tdplotsetrotatedcoords{20}{0}{0};
\draw[tdplot_rotated_coords,axe,color=blue] (O) -- ++(2,0,0) node[anchor=north] {\(x\prim\)};
\draw[tdplot_rotated_coords,axe,color=blue] (O) -- ++(0,2,0) node[anchor=south west] {\(y\prim\)};
\draw[tdplot_rotated_coords,axe,color=blue] (O) -- ++(0,0,2) node[anchor=south west] {\(z\prim\)};
\end{tikzpicture}
\end{center}

On pose \(\omega=\dot{\theta}=\cte\) et le vecteur rotation \(\v{\omega}=\omega\v{u}_z\).

On a \[\v{a}_e=\cancelto{\v{0}}{\odv[order=2]{\v{OO\prim}}{t}_R}+x\prim\odv[order=2]{\v{u}_{x\prim}}{t}_R+y\prim\odv[order=2]{\v{u}_{y\prim}}{t}_R+\cancelto{\v{0}}{z\odv[order=2]{\v{u}_z}{t}_R}\]

On avait \(\odv{\v{u}_{x\prim}}{t}_R=\v{\omega}\vecto\v{u}_{x\prim}\) donc \(\odv[order=2]{\v{u}_{x\prim}}{t}_R=\v{\omega}\vecto\paren{\v{\omega}\vecto\v{u}_{x\prim}}\)

Donc \[\begin{WithArrows}
\v{a}_e&=x\prim\v{\omega}\vecto\paren{\v{\omega}\vecto\v{u}_{x\prim}}+y\prim\v{\omega}\vecto\paren{\v{\omega}\vecto\v{u}_{y\prim}} \Arrow{linéarité de \(\vecto\)} \\
&=\v{\omega}\vecto\paren{\v{\omega}\vecto\paren{x\prim\v{u}_{x\prim}+y\prim\v{u}_{y\prim}}} \Arrow{\(\v{\omega}\vecto\v{u}_z=\v{0}\)} \\
&=\v{\omega}\vecto\paren{\v{\omega}\vecto\paren{x\prim\v{u}_{x\prim}+y\prim\v{u}_{y\prim}+z\v{u}_z}}
\end{WithArrows}\]

D'où \[\v{a}_e=\v{\omega}\vecto\paren{\v{\omega}\vecto\v{OM}}.\]

On se place dans une base cylindrique et on note \(H\) le projeté de \(M\) sur \(\paren{Oz}\) :

\begin{center}
\begin{tikzpicture}[scale=1.6,tdplot_main_coords]
\coordinate (O) at (0,0,0);
\draw[axe] (O) -- ++(2,0,0) node[anchor=north east] {\(x\)};
\draw[axe] (O) -- ++(0,2,0) node[anchor=north west] {\(y\)};
\draw[axe] (O) -- ++(0,0,2) node[anchor=south east] {\(z\)};
\node[left] at (O) {\(O=\textcolor{blue}{O\prim}\)};
\tdplotdrawarc[->]{(O)}{1.5}{0}{20}{anchor=north}{\(\theta\)};
\tdplotsetrotatedcoords{20}{0}{0};
\draw[tdplot_rotated_coords,axe,color=blue] (O) -- ++(2,0,0) node[anchor=north] {\(x\prim\)};
\draw[tdplot_rotated_coords,axe,color=blue] (O) -- ++(0,2,0) node[anchor=south west] {\(y\prim\)};
\draw[tdplot_rotated_coords,axe,color=blue] (O) -- ++(0,0,2) node[anchor=south west] {\(z\prim\)};
\tdplotsetcoord{M}{2.7}{50}{50};
\filldraw (M) circle (1pt);
\node[above right] at (M) {\(M\)};
\draw[dashed] (M) -- (Mz) node[left] {\(H\)};
\draw[dashed] (M) -- (Mxy) -- (O) node[midway, above right] {\(r\)};
\tdplotsetthetaplanecoords{50};
\draw[tdplot_rotated_coords,->,color=blue] (O) -- ++(0,1,0) node[below,blue] {\(\v{u}_r\)};
\draw[tdplot_rotated_coords,->,color=blue] (O) -- ++(0,0,1) node[above left,blue] {\(\v{u}_\theta\)};
\end{tikzpicture}
\end{center}

On a \(\v{OM}=r\v{u}_r+z\v{u}_z=\v{HM}+z\v{u}_z\).

De plus \[\begin{aligned}
\v{a}_e&=\v{\omega}\vecto\paren{\v{\omega}\vecto\paren{r\v{u}_r+z\v{u}_z}} \\
&=\v{\omega}\vecto\paren{\v{\omega}\vecto r\v{u}_r} \\
&=\v{\omega}\vecto\omega r\v{u}_\theta \\
&=-\omega^2r\v{u}_r
\end{aligned}\]

Donc \[\textcolor{red}{\v{a}_e=-\omega^2\v{HM}}\] où \(H\) est le projeté de \(M\) sur l'axe de rotation.

Remarques : \begin{itemize}
    \item \(\v{a}_e\) dépend de la position de \(M\) dans \(R\prim\) et de \(\norme{\v{HM}}\) (distance à l'axe de rotation). \\
    \item \(\v{a}_e\) est dirigée vers l'axe de rotation (accélération centripète). \\
\end{itemize}

Déterminons maintenant \(\v{a}_C\) : \[\begin{WithArrows}
\v{a}_C&=2\paren{\odv{x\prim}{t}\odv{\v{u}_{x\prim}}{t}_R+\odv{y\prim}{t}\odv{\v{u}_{y\prim}}{t}_R+\odv{z\prim}{t}\cancelto{\v{0}}{\odv{\v{u}_{z\prim}}{t}_R}} \\
&=2\paren{\odv{x\prim}{t}\v{\omega}\vecto\v{u}_{x\prim}+\odv{y\prim}{t}\v{\omega}\vecto\v{u}_{y\prim}} \Arrow{linéarité de \(\vecto\)} \\
&=2\v{\omega}\vecto\paren{\odv{x\prim}{t}\v{u}_{x\prim}+\odv{y\prim}{t}\v{u}_{y\prim}} \Arrow{\(\v{\omega}\vecto\v{u}_{z\prim}=\v{0}\)} \\
&=2\v{\omega}\vecto\paren{\odv{x\prim}{t}\v{u}_{x\prim}+\odv{y\prim}{t}\v{u}_{y\prim}+\odv{z\prim}{t}\v{u}_{z\prim}}
\end{WithArrows}\]

D'où \[\textcolor{red}{\v{a}_C=2\v{\omega}\vecto\v{v}\paren{M}_{R\prim}}\]

Remarques : \begin{itemize}
    \item Si \(\v{v}\paren{M}_{R\prim}=\v{0}\) alors \(\v{a}_C=\v{0}\). \\
    \item La direction de \(\v{a}_C\) dépend de \(\v{v}\paren{M}_{R\prim}\). \\
\end{itemize}

Exercice/exemple : calculons l'accélération d'entraînement du référentiel terrestre par rapport au référentiel géocentrique à Limoges (latitude \(\lambda=\SI{49}{\degree}N\)).

\begin{center}
\tdplotsetmaincoords{80}{90}
\begin{tikzpicture}[scale=2,tdplot_main_coords]
\coordinate (O) at (0,0,0);
\draw[thick] (O) -- ++(0,2,0);
\draw[thick] (O) -- ++(0,0,2) node[anchor=south] {\(z\) : axe des pôles};
\node[left] at (O) {\(O\)};
\begin{scope}[tdplot_screen_coords]
\fill[ball color=gray!20, opacity=0.25] (O) circle (1.5);
\end{scope}
\draw (O) circle[radius=1.5];
\draw[dashed] (0,0,0.99) circle[radius=1.13];
\tdplotsetthetaplanecoords{90};
\tdplotdrawarc[tdplot_rotated_coords]{(O)}{0.4}{90}{49}{anchor=west}{\(\lambda\)};
\tdplotsetcoord{L}{1.5}{49}{90};
\draw[dashed,red] (O) -- (L) node[anchor=south west] {\(L\)};
\filldraw[red] (L) circle (1pt);
\draw[dashed] (L) -- (Lz) node[anchor=east] {\(H\)};
\node[anchor=north] at (2,0,0) {plan équatorial};
\draw[thick] (L) -- ++(0,0.5,0);
\draw[thick] (L) -- ++(0,0,0.5);
\end{tikzpicture}
\end{center}

Le référentiel terrestre est en rotation uniforme par rapport au référentiel géocentrique : \(\v{\omega}=\omega\v{u}_z\).

\(H\) est le projeté de \(L\) sur \(Oz\).

On a \(\v{a}_e=-\omega^2\v{HL}\) donc \(\norme{\v{a}_e}=\omega^2HL\).

Or \(\omega=\dfrac{2\pi}{T}\) avec \(T\) la durée du jour sidéral (\(\SI{23}{\hour}\) et \(\SI{56}{\min}\)).

Donc \(\omega=\SI{7.29e-5}{\radian\per\second}\).

On note \(R_T=\SI{6400}{\kilo\metre}\) le rayon de la Terre.

On a \(HL=R_T\cos\lambda\).

Donc \(a_e=\SI{2.7e-2}{\metre\per\second\squared}\).

On compare à \(g=\SI{10}{\metre\per\second\squared}\) : \(\dfrac{a_e}{g}=\num{2.7e-3}\ll1\).

Il est donc difficile de mettre en évidence l'effet de \(\v{a}_e\) du référentiel terrestre par rapport au référentiel géocentrique.

Remarque : l'accélération d'entraînement est nulle aux pôles et maximale à l'équateur.

\section{Notion de forces d'inertie}

\subsection{Cas général}

Soient \(R_g\) un référentiel galiléen et \(R\) un référentiel d'étude quelconque.

On a \[\begin{dcases}
\v{v}\paren{M}_{R_g}=\v{v}\paren{M}_R+\v{v}_e \\
\v{a}\paren{M}_{R_g}=\v{a}\paren{M}_R+\v{a}_e+\v{a}_C
\end{dcases}\]

Soit un point \(M\) de masse \(m\) soumis à un ensemble de forces \(\v{F}\) dans \(R_g\).

Comme \(R_g\) est galiléen, on applique le principe fondamental de la dynamique : \(m\v{a}\paren{M}_{R_g}=\v{F}\).

On applique la loi de composition des accélérations et on obtient : \(m\paren{\v{a}\paren{M}_R+\v{a}_e+\v{a}_C}=\v{F}\), \cad : \[m\v{a}\paren{M}_R=\v{F}-m\v{a}_e-m\v{a}_C.\]

Donc dans un référentiel non-galiléen, le principe fondamental de la dynamique s'écrit avec deux termes supplémentaires :

\begin{itemize}
    \item \(\v{F}_{ie}=-m\v{a}_e\) : force d'inertie d'entraînement ; \\
    \item \(\v{F}_{iC}=-m\v{a}_C\) : force d'inertie de Coriolis. \\
\end{itemize}

Remarque : \(\v{F}\) sont des forces véritables, elles ont une origine physique ; \(\v{F}_{ie}\) et \(\v{F}_{iC}\) sont des pseudo-forces dont l'origine est le caractère non-galiléen de \(R\).

Remarque : on a \(\v{F}_{iC}=-2m\v{\omega}\vecto\v{v}\paren{M}_R\) donc \(\v{F}_{iC}\not=\v{0}\) si \(R\) est en rotation par rapport à \(R_g\) et si \(M\) est en mouvement dans \(R\).

\subsection{Cas de \(R\) en translation par rapport à \(R_g\)}

Soit un point \(M\) de masse \(m\) accroché au plafond d'un ascenseur en mouvement vertical rectiligne.

\begin{center}
\tdplotsetmaincoords{70}{110}
\begin{tikzpicture}[scale=1.2,tdplot_main_coords]
\coordinate (O) at (0,0,0);
\draw[thick,->] (O) -- ++(1,0,0);
\draw[thick,->] (O) -- ++(0,1,0);
\draw[thick,->] (O) -- ++(0,0,1);
\node[above left] at (O) {\(R\)};
\draw (O) -- ++(0,3,0) -- ++(0,0,-4) -- ++(0,-3,0) -- ++(0,0,4);
\draw (0,1.5,0) -- ++(0,0,-1.5) coordinate (M) node[right] {\(M\)};
\filldraw (M) circle (1pt);
\draw[red,->] (M) -- ++(0,0,-1) node[left] {\(\v{P}\)};
\draw[red,->] (M) -- ++(0,0,1) node[left] {\(\v{T}\)};
\draw[green, ->] (0,1.5,0) -- ++(0,0,1.5) node[left] {\(\v{v}_e\)};
\draw[violet,->] (0,4,2) -- ++(0,0,-1) node[left] {\(\v{g}\)};
\coordinate (O') at (0,0,-6);
\draw[thick,->] (O') -- ++(1,0,0);
\draw[thick,->] (O') -- ++(0,1,0);
\draw[thick,->] (O') -- ++(0,0,1);
\node[above left] at (O') {\(R_g\)};
\end{tikzpicture}
\end{center}

Système : \(M\paren{m}\).

Référentiel : \(R\) non-galiléen.

Bilan des forces : \(\v{P}\) et \(\v{T}\).

\PFD : \(m\v{a}=\v{P}+\v{T}-m\v{a}_e\) (\(\v{F}_{iC}=\v{0}\)) où \begin{description}
    \item[] \(\v{a}_e\) est l'accélération d'entraînement de l'ascenseur par rapport à \(R_g\) \\
    \item[] \(\v{a}\) est l'accélération de \(M\) dans \(R\) donc ici \(\v{a}=\v{0}\) \\
    \item[] \(\v{P}=m\v{g}\). \\
\end{description}

Donc \(\v{0}=m\v{g}+\v{T}-m\v{a}_e\)

Donc \(\v{T}=m\paren{\v{a}_e-\v{g}}\).

On définit alors le poids apparent \(\v{P}\prim=-\v{T}=m\paren{\v{g}-\v{a}_e}=m\v{g}\prim\) où \(\v{g}\prim\) est le champ de pesanteur apparent.

Si l'ascenseur accélère vers le haut alors \(g\prim=g-a_e>g\).

Si l'ascenseur accélère vers le bas alors \(g\prim=g-a_e<g\).

Si l'ascenseur est en chute libre alors \(\v{g}=\v{a}_e\) donc \(\v{g}\prim=\v{0}\) : on est en chute libre dans l'ascenseur.

Conclusion : travailler dans un référentiel non-galiléen qui est en translation rectiligne par rapport au référentiel terrestre (galiléen) revient à remplacer \(\v{g}\) par \(\v{g}\prim=\v{g}-\v{a}_e\).

\attention \(\v{g}\) et \(\v{a}_e\) ne sont pas forcément colinéaires.

Exemple : voiture qui accélère.

\begin{center}
\begin{tikzpicture}
\draw (0,0) -- (0,1) -- (2,1) -- (3,2) -- (5,2) -- (6,1) -- (8,1) -- (8,0) -- (0,0);
\draw (2,-0.5) circle (0.5);
\draw (6,-0.5) circle (0.5);
\draw[green,->] (7,1.5) -- (9,1.5) node[above] {\(\v{v}_e\)};
\draw[green,->] (7,-1.5) -- (9,-1.5) node[below] {\(\v{a}_e\)};
\coordinate (O) at (5.5,1.5);
\draw (O) -- ++(225:1) coordinate (M);
\draw[dashed] (O) -- ++(0,-1) coordinate (A);
\pic[draw,<-,"\(\alpha\)",angle eccentricity=1.5] {angle=M--O--A};
\filldraw (M) circle (1pt);
\draw[violet,->] (M) -- ++(0,-2) coordinate (B) node[right] {\(\v{g}\)};
\draw[green,->] (B) -- ++(-2,0) coordinate (C) node[below] {\(-\v{a}_e\)};
\draw[violet,->] (M) -- (C) node[above] {\(\v{g}\prim\)};
\pic[draw,<-,"\(\alpha\)",angle eccentricity=2] {angle=C--M--B};
\end{tikzpicture}
\end{center}

Avec \(\alpha=\Arctan\dfrac{a_e}{g}\).

\subsection{Cas de \(R\) en rotation uniforme autour d'un axe fixe de \(R_g\)}

\begin{center}
\begin{tikzpicture}[scale=1.6,tdplot_main_coords]
\coordinate (O) at (0,0,0);
\draw[axe] (O) -- ++(2,0,0) node[anchor=north east] {\(x\)};
\draw[axe] (O) -- ++(0,2,0) node[anchor=north west] {\(y\)};
\draw[axe] (O) -- ++(0,0,2) node[anchor=south east] {\(z\)};
\node[left] at (O) {\(O=\textcolor{blue}{O\prim}\)};
\tdplotdrawarc[->]{(O)}{1.5}{0}{20}{anchor=north}{\(\theta\)};
\tdplotsetrotatedcoords{20}{0}{0};
\draw[tdplot_rotated_coords,axe,color=blue] (O) -- ++(2,0,0) node[anchor=north] {\(x\prim\)};
\draw[tdplot_rotated_coords,axe,color=blue] (O) -- ++(0,2,0) node[anchor=south west] {\(y\prim\)};
\draw[tdplot_rotated_coords,axe,color=blue] (O) -- ++(0,0,2) node[anchor=south west] {\(z\prim\)};
\tdplotsetcoord{M}{2.7}{50}{50};
\filldraw (M) circle (1pt);
\node[above right] at (M) {\(M\)};
\draw[dashed] (M) -- (Mz) node[left] {\(H\)};
\draw[dashed] (M) -- (Mxy) -- (O) node[midway, above right] {\(r\)};
\tdplotsetthetaplanecoords{50};
\draw[tdplot_rotated_coords,->,color=blue] (O) -- ++(0,1,0) node[below,blue] {\(\v{u}_r\)};
\draw[tdplot_rotated_coords,->,color=blue] (O) -- ++(0,0,1) node[above left,blue] {\(\v{u}_\theta\)};
\end{tikzpicture}
\end{center}

On pose \(\v{\omega}=\dot{\theta}\v{u}_z\).

On a \(\v{a}_e=\v{\omega}\vecto\paren{\v{\omega}\vecto\v{OM}}=-\omega^2\v{HM}\) avec \(H\) le projeté de \(M\) sur \(\paren{Oz}\).

Donc \(\v{F}_{ie}=-m\v{a}_e=m\omega^2\v{HM}\) (effet centrifuge).

Exemple : pendule conique.

Soit un point \(M\) de masse \(m\) attaché à un fil de longueur \(l\).

\begin{center}
\begin{tikzpicture}[scale=1.6,tdplot_main_coords]
\coordinate (C) at (0,0,0);
\draw[axe] (C) -- ++(2,0,0) node[anchor=north east] {\(x\)};
\draw[axe] (C) -- ++(0,2,0) node[anchor=north west] {\(y\)};
\draw[axe] (C) -- ++(0,0,2) node[anchor=south east] {\(z\)};
\tdplotdrawarc[->]{(C)}{1.5}{0}{50}{anchor=north}{\(\theta\)};
\tdplotsetrotatedcoords{50}{0}{0};
\draw[tdplot_rotated_coords,axe,color=blue] (C) -- ++(2,0,0) node[anchor=north east] {\(x\prim\)};
\draw[tdplot_rotated_coords,axe,color=blue] (C) -- ++(0,2,0) node[anchor=south west] {\(y\prim\)};
\draw[tdplot_rotated_coords,axe,color=blue] (C) -- ++(0,0,2) node[anchor=south west] {\(z\prim\)};
\tdplotsetcoord{M}{2.5}{90}{50};
\filldraw (M) circle (1pt);
\node[above right] at (M) {\(M\)};
\coordinate (O) at (0,0,1.5);
\filldraw (O) circle (1pt);
\node[anchor=south east] at (O) {\(O\)};
\draw[thick] (O) -- (M) node[pos=0.35,right] {\(l\)};
\tdplotdefinepoints(0,0,1.5)(0,0,0)(1.607,1.915,0);
\tdplotdrawpolytopearc{0.5}{anchor=north}{\(\alpha\)};
\draw[->,red] (M) -- ++(0,0,-0.981) node[anchor=east] {\(\v{P}\)};
\tdplotsetrotatedcoords{50}{0}{0};
\draw[->,red,tdplot_rotated_coords] (M) -- ++(0.617,0,0) node[anchor=north west] {\(\v{F}_{ie}\)};
\draw[->,red,tdplot_rotated_coords] (M) -- ++(-0.857,0,0.514) node[anchor=east] {\(\v{T}\)};
\draw[->,violet] (2,0,1.5) -- ++(0,0,-1) node[anchor=east] {\(\v{g}\)};
\draw[dashed] (C) circle (2.5);
\node[anchor=east] (C) {\(H\)};
\end{tikzpicture}
\end{center}

\(M\) est en rotation autour de la verticale.

Le référentiel \(\paren{H,x,y,z}\) est galiléen et le référentiel \(R=\paren{H,x\prim,y\prim,z\prim}\) est non-galiléen et en rotation autour de \(\paren{Hz}\).

Système : \(M\paren{m}\).

Référentiel : \(R\) non-galiléen.

Bilan des forces : \(\v{P}=m\v{g}\) et \(\v{T}\).

\PFD : \(m\v{a}=\v{P}+\v{T}+\v{F}_{ie}+\v{F}_{iC}\).

Or \(\v{F}_{iC}=\v{0}\) car \(\v{v}\paren{M}_R=\v{0}\) et \(\v{F}_{ie}=m\omega^2\v{HM}\).

De plus, on a \(HM=l\sin\alpha\) donc \(\v{F}_{ie}=m\omega^2l\sin\alpha\v{u}_r\).

Or \(M\) est fixe dans \(R\) donc \(\v{a}=\v{0}\).

Donc \(\v{0}=\v{P}+\v{T}+\v{F}_{ie}\).

On a : \[\begin{aligned}
\tan\alpha&=\dfrac{\norme{\v{F}_{ie}}}{\norme{\v{P}}} \\
\dfrac{\sin\alpha}{\cos\alpha}&=\dfrac{m\omega^2l\sin\alpha}{mg} \\
\omega^2&=\dfrac{g}{l\cos\alpha}>\dfrac{g}{l} \\
\omega&>\sqrt{\dfrac{g}{l}}.
\end{aligned}\]

Le pendule conique ne peut pas tourner trop lentement : \[f_\mini=\dfrac{\omega_\mini}{2\pi}=\dfrac{1}{2\pi}\sqrt{\dfrac{g}{l}}.\]

Application numérique : avec \(l=\SI{1}{\metre}\), on a \(f_\mini=\SI{0.5}{\hertz}\).

Déterminons maintenant si \(\v{F}_{ie}\) et \(\v{F}_{iC}\) sont conservatives.

On a \(\v{F}_{ie}=m\omega^2\v{HM}=m\omega^2r\v{u}_r\) et \(\odif{\v{OM}}=\odif{r}\v{u}_r+r\odif{\theta}\v{u}_\theta+\odif{z}\v{u}_z\) donc \[\begin{aligned}
\fdif{W}&=\v{F}_{ie}\scal\odif{\v{OM}} \\
&=m\omega^2r\v{u}_r\scal\paren{\odif{r}\v{u}_r+r\odif{\theta}\v{u}_\theta+\odif{z}\v{u}_z} \\
&=m\omega^2r\odif{r} \\
&=-\odif{\paren{\dfrac{-1}{2}m\omega^2r^2+\cte}} \\
&=-\odif{E_p}
\end{aligned}\] avec \(E_p=\dfrac{-1}{2}m\omega^2r^2+\cte\).

Donc \(\v{F}_{ie}\) est conservative.

De plus, on a \(\v{F}_{iC}=-2m\v{\omega}\vecto\v{v}\paren{M}_R\perp\v{v}\paren{M}_R\) et \(\odif{\v{OM}}=\odv{\v{OM}}{t}\odif{t}=\v{v}\paren{M}_R\odif{t}\) donc \[\begin{aligned}
\fdif{W}&=\v{F}_{iC}\scal\odif{\v{OM}} \\
&=0.
\end{aligned}\]

Donc \(\v{F}_{iC}\) ne travaille pas et n'a pas d'énergie potentielle associée.

Reprenons l'exemple du pendule conique :

\begin{center}
\begin{tikzpicture}[scale=1.6,tdplot_main_coords]
\coordinate (C) at (0,0,0);
\draw[axe] (C) -- ++(2,0,0) node[anchor=north east] {\(x\)};
\draw[axe] (C) -- ++(0,2,0) node[anchor=north west] {\(y\)};
\draw[axe] (C) -- ++(0,0,2) node[anchor=south east] {\(z\)};
\tdplotdrawarc[->]{(C)}{1.5}{0}{50}{anchor=north}{\(\theta\)};
\tdplotsetrotatedcoords{50}{0}{0};
\draw[tdplot_rotated_coords,axe,color=blue] (C) -- ++(2,0,0) node[anchor=north east] {\(x\prim\)};
\draw[tdplot_rotated_coords,axe,color=blue] (C) -- ++(0,2,0) node[anchor=south west] {\(y\prim\)};
\draw[tdplot_rotated_coords,axe,color=blue] (C) -- ++(0,0,2) node[anchor=south west] {\(z\prim\)};
\tdplotsetcoord{M}{2.5}{90}{50};
\filldraw (M) circle (1pt);
\node[above right] at (M) {\(M\)};
\coordinate (O) at (0,0,1.5);
\filldraw (O) circle (1pt);
\node[anchor=south east] at (O) {\(O\)};
\draw[thick] (O) -- (M) node[pos=0.35,right] {\(l\)};
\tdplotdefinepoints(0,0,1.5)(0,0,0)(1.607,1.915,0);
\tdplotdrawpolytopearc{0.5}{anchor=north}{\(\alpha\)};
\draw[->,red] (M) -- ++(0,0,-0.981) node[anchor=east] {\(\v{P}\)};
\tdplotsetrotatedcoords{50}{0}{0};
\draw[->,red,tdplot_rotated_coords] (M) -- ++(0.617,0,0) node[anchor=north west] {\(\v{F}_{ie}\)};
\draw[->,red,tdplot_rotated_coords] (M) -- ++(-0.857,0,0.514) node[anchor=east] {\(\v{T}\)};
\draw[->,violet] (2,0,1.5) -- ++(0,0,-1) node[anchor=east] {\(\v{g}\)};
\draw[dashed] (C) circle (2.5);
\node[anchor=east] (C) {\(H\)};
\end{tikzpicture}
\end{center}

\(\v{P}\), \(\v{T}\) et \(\v{F}_{iC}\) ne travaillent pas et \(\v{F}_{ie}\) est conservative.

La variable géométrique est \(\alpha\).

Dans le référentiel \(R\) tournant non-galiléen, on a \(E_m=E_c+E_{p_\mathrm{pesanteur}}+E_{p_{\v{F}_{ie}}}\).

Comme \(v=l\dot{\alpha}\), on a : \[E_c=\dfrac{1}{2}ml^2\dot{\alpha}^2\] et \[E_{p_\mathrm{pesanteur}}=-mgl\cos\alpha\] et \[\begin{aligned}
E_{p_{\v{F}_{ie}}}&=\dfrac{-1}{2}m\omega^2r^2+\cancelto{0}{\cte} \\
&=\dfrac{-1}{2}m\omega^2l^2\sin^2\alpha.
\end{aligned}\]

Donc \(E_m=\dfrac{1}{2}ml^2\dot{\alpha}^2-mgl\cos\alpha-\dfrac{1}{2}m\omega^2l^2\sin^2\alpha\).

Or comme \(M\) n'est soumis à aucune force conservative, on a \(E_m=\cte\).

Donc \[\begin{aligned}
\odv{E_m}{t}&=0 \\
ml^2\dot{\alpha}\ddot{\alpha}+mgl\dot{\alpha}\sin\alpha-ml^2\omega^2\dot{\alpha}\sin\alpha\cos\alpha&=0 \\
l\ddot{\alpha}+g\sin\alpha-l\omega^2\sin\alpha\cos\alpha&=0
\end{aligned}\]

C'est l'équation du mouvement du pendule conique dans \(R\) autour de sa position d'équilibre.

Si on considère une position d'équilibre dans \(R\), \ie \(\alpha=\cte\), on retrouve \(\omega^2=\dfrac{g}{l\cos\alpha}\).

\section{Caractère galiléen approché du référentiel terrestre}

\subsection{Les référentiels d'étude}

Rappel : un référentiel \(R\prim\) est galiléen s'il est en translation rectiligne uniforme par rapport à un référentiel galiléen. En effet, dans ce cas, on a \(\v{F}_{ie}=\v{0}\) et \(\v{F}_{iC}=\v{0}\) donc d'après le \PFD dans \(R\prim\), on a \(m\v{a}=\v{F}\) donc le principe d'inertie est vérifié dans \(R\prim\) donc \(R\prim\) est galiléen.

\subsubsection{Référentiel de Copernic}

C'est le référentiel d'origine \(C\) le centre de masse du système solaire et d'axes \(\paren{Cx_g,Cy_g,Cz_g}\) dirigés vers trois étoiles lointaines considérées comme fixes car suffisamment éloignées du système solaire.

Le référentiel de Copernic est le référentiel galiléen de base. Les observations astronomiques ont montré qu'une météorite se déplaçant dans l'espace suffisamment éloignée des différentes planètes a bien un mouvement de translation rectiligne uniforme.

\subsubsection{Référentiel de Kepler}

C'est le référentiel d'origine \(S\) le centre de masse du soleil et d'axes \(\paren{Sx_0,Sy_0,Sz_0}\) parallèles à \(\paren{Cx_g,Cy_g,Cz_g}\) dirigés vers trois étoiles lointaines considérées comme fixes car suffisamment éloignées du système solaire.

En pratique, le référentiel de Kepler est quasi-galiléen et il est très difficile de mettre en évidence son caractère non-galiléen.

\subsubsection{Référentiel géocentrique}

C'est le référentiel d'origine \(O\) le centre de masse de la Terre et d'axes \(\paren{Ox_1,Oy_1,Oz_1}\) dirigés vers trois étoiles lointaines considérées comme fixes car suffisamment éloignées du système solaire.

Si une expérience dure peu de temps devant une année, on peut considérer que la Terre se déplace en ligne droite sur son orbite et donc que le référentiel géocentrique est galiléen.

\subsubsection{Référentiel terrestre local}

Le référentiel terrestre local \(R_T\) est constitué d'une origine \(A\) liée au sol et à trois axes \(\paren{Ax,Ay,Az}\). Il est en rotation uniforme autour de l'axe des pôles de la Terre donc n'est pas galiléen.

Si une expérience est courte devant vingt-quatre heures, on peut considérer que \(R_T\) se déplace en ligne droite par rapport au référentiel géocentrique et donc que son caractère non-galiléen est peu marqué.

\begin{center}
\tdplotsetmaincoords{80}{110}
\begin{tikzpicture}[scale=2,tdplot_main_coords]
\coordinate (O) at (0,0,0);
\draw[thick] (O) -- ++(2,0,0) node[anchor=north east] {\(x_1\)};
\draw[thick] (O) -- ++(0,2,0) node[anchor=north west] {\(y_1\)} node[above right] {\(R_\geo\)};
\draw[thick] (O) -- ++(0,0,2) node[anchor=south] {\(z_1\)};
\node[anchor=south east] at (O) {\(O\)};
\begin{scope}[tdplot_screen_coords]
\fill[ball color=gray!20, opacity=0.25] (O) circle (1.5);
\end{scope}
\node[above right] at (0,0,1.5) {\(N\)};
\draw[dashed] (O) -- ++(0,0,-1.5) node[below right] {\(S\)};
\draw[dashed,blue,decoration={markings,mark=at position 0.65 with {\arrow{>}}},postaction={decorate}] (0,0,0.99) circle[radius=1.13];
\tdplotsetrotatedcoords{90}{49}{90};
\tdplotsetcoord{A}{1.5}{49}{90};
\filldraw[blue] (A) circle (1pt);
\draw[thick,blue,tdplot_rotated_coords] (A) -- ++(0,-0.5,0);
\draw[thick,blue,tdplot_rotated_coords] (A) -- ++(0,0,0.5) node[right] {\(R_T\) en rotation autour de \(R_\geo\)};
\draw[thick,blue,tdplot_rotated_coords] (A) -- ++(-0.5,0,0);
\end{tikzpicture}
\end{center}

\subsection{Mécanique dans le référentiel terrestre}

\subsubsection{Statique terrestre, champ de pesanteur}

On se place dans le référentiel terrestre local non-galiléen.

Soit \(M\paren{m}\) accroché au plafond à l'équilibre.

\begin{center}
\begin{tikzpicture}
\draw[ultra thick] (0,0) -- ++(4,0);
\fill[pattern=north east lines] (0,0) -- ++(4,0) -- ++(0,0.5) -- ++(-4,0);
\draw (2,0) -- ++(0,-2.5) coordinate (M) node[left] {\(M\)};
\fill (M) circle (2pt);
\draw[->,red] (M) -- ++(0,1) node[right] {\(\v{T}\)};
\draw[->,red] (M) -- ++(0,-1.5) node[right] {\(\v{F}_\grav\)};
\draw[->,red] (M) -- ++(1.3,0.8) node[right] {\(\v{F}_{ie}\)};
\end{tikzpicture}
\end{center}

Système : \(M\paren{m}\).

Référentiel : terrestre non-galiléen.

Bilan des forces : \(\v{F}_\grav=-\dfrac{GmM_T}{R_T^2}\v{u}\) avec \(\v{u}=\dfrac{\v{OM}}{R_T}\), \(\v{T}\), \(\v{F}_{ie}=m\omega^2\v{HM}\) et \(\v{F}_{iC}=\v{0}\).

\PFD : \(\v{F}_\grav+\v{T}+\v{F}_{ie}=\v{0}\).

On définit \[\begin{aligned}
\v{P}&=-\v{T} \\
&=\v{F_\grav}+\v{F}_{ie} \\
&=-\dfrac{GmM_T}{R_T^2}+m\omega^2\v{HM} \\
&=m\paren{-\dfrac{GM_T}{R_T^2}+\omega^2\v{HM}}.
\end{aligned}\]

On pose alors le champ gravitationnel \(\v{\call{G}}_\grav=-\dfrac{GM_T}{R_T^2}\).

On obtient le champ de pesanteur \[\v{g}=\v{\call{G}}_\grav\paren{M}+\omega^2\v{HM}.\]

Ainsi, dans le référentiel terrestre local, la force d'inertie d'entraînement est prise en compte dans la définition du poids.

Remarques :

\begin{itemize}
    \item \(\v{g}\) n'est pas colinéaire à \(\v{OM}\). \\
    \item La direction de \(\v{g}\) définit la verticale. \\
    \item On a \(\norme{\v{g}}\not=\norme{\v{\call{G}}_\grav}\). \\
    \item On a \(g_\pole=\dfrac{GM_T}{R_T^2}\) (en considérant la Terre sphérique). \\
    \item On a \(g_\equateur=\dfrac{GM_T}{R_T^2}-\omega^2R_T\). \\
    \item On a \(\dfrac{\adif{g}}{g_\pole}=\dfrac{\omega^2R_T}{G\frac{M_T}{R_T^2}}=\num{3e-3}\).
\end{itemize}

\subsubsection{Dynamique terrestre}

\paragraph{Base locale}~\\

Dans le référentiel terrestre local, on ne tient pas compte de \(\v{F}_{ie}=m\omega^2\v{HM}\) car elle est déjà contenue dans le poids.

\begin{center}
\tdplotsetmaincoords{80}{110}
\begin{tikzpicture}[scale=2,tdplot_main_coords]
\coordinate (O) at (0,0,0);
\draw[thick] (O) -- ++(2,0,0);
\draw[thick] (O) -- ++(0,2,0);
\draw[thick] (O) -- ++(0,0,2);
\draw[thick,Goldenrod,->] (O) -- ++(0,0,0.8) node[left] {\(\v{\Omega}\)};
\begin{scope}[tdplot_screen_coords]
\fill[ball color=gray!20, opacity=0.25] (O) circle (1.5);
\end{scope}
\node[above right] at (0,0,1.5) {\(N\)};
\draw[dashed] (O) -- ++(0,0,-1.5) node[below right] {\(S\)};
\tdplotsetrotatedcoords{90}{49}{90};
\tdplotsetcoord{A}{1.5}{49}{90};
\draw[dashed] (O) -- (A);
\filldraw[blue] (A) circle (1pt);
\draw[thick,blue,tdplot_rotated_coords,->] (A) -- ++(0.5,0,0) node[below right] {\(\v{u}_x\)};
\draw[thick,blue,tdplot_rotated_coords,->] (A) -- ++(0,0.5,0) node[above right] {\(\v{u}_y\)};
\draw[thick,blue,tdplot_rotated_coords,->] (A) -- ++(0,0,0.5) node[above right] {\(\v{u}_z\)};
\tdplotsetthetaplanecoords{90};
\tdplotdrawarc[tdplot_rotated_coords]{(O)}{0.4}{90}{49}{anchor=west}{\(\lambda\)};
\end{tikzpicture}
\end{center}

On a la vitesse de rotation de la Terre autour de son axe \(\Omega=\SI{7.3e-5}{\radian\per\second}\).

On pose le vecteur rotation \(\v{\Omega}=\Omega\dfrac{\v{SN}}{R_T}\).

Base locale à la surface de la Terre : \(\v{u}_x\) vers l'Est, \(\v{u}_y\) vers le Nord et \(\v{u}_z\) vertical ascendant.

\attention Base sphérique \(\paren{\v{u}_r,\v{u}_\theta,\v{u}_\phi}=\paren{\v{u}_z,-\v{u}_y,\v{u}_x}\).

On sait que \(\v{F}_{ie}\) est déjà contenue dans \(\v{P}\).

On note \(\v{v}=\dot{x}\v{u}_x+\dot{y}\v{u}_y+\dot{z}\v{u}_z\) la vitesse dans le référentiel terrestre local.

On a \(\v{\Omega}=\Omega\paren{\cos\paren{\lambda}\v{u}_y+\sin\paren{\lambda}\v{u}_z}\).

Donc \[\begin{aligned}
\v{F}_{iC}&=-2m\v{\Omega}\vecto\v{v} \\
&=-2m\Omega\paren{\dot{x}\paren{\cos\paren{\lambda}\v{u}_z-\sin\paren{\lambda}\v{u}_y}+\dot{y}\sin\paren{\lambda}\v{u}_x-\dot{z}\cos\paren{\lambda}\v{u}_x} \\
&=-2m\Omega\paren{\paren{\dot{y}\sin\lambda-\dot{z}\cos\lambda}\v{u}_x-\dot{x}\sin\paren{\lambda}\v{u}_y+\dot{x}\cos\paren{\lambda}\v{u}_z}.
\end{aligned}\]

\paragraph{Mouvement horizontal, vers la droite dans l'hémisphère Nord}~\\

On considère un mouvement horizontal donc \(z=\cte\).

On a \(\dot{x}\not=0\), \(\dot{y}\not=0\) et \(\dot{z}=0\).

Donc \[\begin{aligned}
\v{F}_{iC}&=-2m\v{\Omega}\vecto\v{v} \\
&=-2m\Omega\paren{\cos\paren{\lambda}\v{u}_y+\sin\paren{\lambda}\v{u}_z}\vecto\v{v} \\
&=-2m\Omega\paren{\underbrace{\cos\paren{\lambda}\v{u}_y\vecto\v{v}}_{\text{force portée par }\v{u}_z}+\underbrace{\sin\paren{\lambda}\v{u}_z\vecto\v{v}}_{\text{force portée par }\v{u}_x}}.
\end{aligned}\]

Ainsi, la force de Coriolis tend à pousser un mobile vers la droite du mouvement dans l'hémisphère Nord (\(\lambda>0\)) et vers la gauche du mouvement dans l'hémisphère Sud (\(\lambda<0\)).

Ordre de grandeur : en considérant une voiture de \(m=\SI{1000}{\kilo\gram}\) roulant à \(v=\SI{100}{\kilo\metre\per\hour}\) avec \(\lambda=\SI{49}{\degree}\), on obtient \[F=2m\Omega v\sin\lambda\approx\SI{3}{\newton}.\] Donc l'effet de la force de Coriolis n'est sensible que pour des objets de masse importante et de vitesse élevée.

\paragraph{Mouvement vertical, déviation vers l'Est}~\\

Deux expériences ont montré l'existence d'une déviation de la trajectoire d'un objet en chute libre :

\begin{itemize}
    \item Expérience de Reich (1831) : chute de billes d'acier dans des puits de mine (sans vent). Avec \(h=\SI{158}{m}\) et \(\lambda=\ang{50}\), on observa une déviation de \(\SI{28}{\milli\metre}\) par rapport à la verticale. \\
    \item Expérience de Flammarion (1903) : idem depuis la coupole du Panthéon. Avec \(h=\SI{68}{\metre}\) et \(\lambda=\ang{48;51}\), on observa une déviation de \(\SI{7.6}{\milli\metre}\) par rapport à la verticale.
\end{itemize}

Système : bille de masse \(m\).

Référentiel : terrestre supposé galiléen (première approche sans \(\v{F}_{iC}\)).

Bilan des forces : \(\v{P}=m\v{g}=-mg\v{u}_z\) (contient \(\v{F}_{ie}\)).

\PFD : \(m\tcoords{\ddot{x}}{\ddot{y}}{\ddot{z}}=-mg\tcoords{0}{0}{1}\)

D'où \[\begin{dcases}
\ddot{x}=0 \\
\ddot{y}=0 \\
\ddot{z}=-g
\end{dcases}\ssi\begin{dcases}
\dot{x}=0 \\
\dot{y}=0 \\
\dot{z}=-gt
\end{dcases}\ssi\begin{dcases}
x=0 \\
y=0 \\
z=h-\dfrac{1}{2}gt^2
\end{dcases}\]

Donc sans tenir compte de \(\v{F}_{iC}\), on obtient le mouvement non-perturbé (ou mouvement d'ordre 0) avec \[\v{v}=-gt\v{u}_z.\]

On a \[\begin{aligned}
\v{F}_{iC}&=-2m\v{\Omega}\vecto\v{v} \\
&=-2m\Omega\paren{\cos\paren{\lambda}\v{u}_y+\sin\paren{\lambda}\v{u}_z}\vecto\paren{-gt\v{u}_z} \\
&=2m\Omega\cos\paren{\lambda}gt\v{u}_x.
\end{aligned}\]

Méthode perturbative : on sait que \(\norme{\v{F}_{iC}}\ll mg\) ; on considère que l'ordre 0 reste vrai et on le perturbe avec \(\v{F}_{iC}\).

Bilan des forces : \(\v{P}=-mg\v{u}_z\) et \(\v{F}_{iC}=2m\Omega\cos\paren{\lambda}gt\v{u}_x\).

\PFD : \(m\tcoords{\ddot{x}}{\ddot{y}}{\ddot{z}}=\tcoords{2m\Omega\cos\paren{\lambda}gt}{0}{-mg}\).

D'où \[\begin{dcases}
\ddot{x}=2\Omega\cos\paren{\lambda}gt \\
\ddot{y}=0 \\
\ddot{z}=-g
\end{dcases}\ssi\begin{dcases}
\dot{x}=\Omega\cos\paren{\lambda}gt^2 \\
\dot{y}=0 \\
\dot{z}=-gt
\end{dcases}\ssi\begin{dcases}
x=\dfrac{1}{3}\Omega\cos\paren{\lambda}t^3 \\
y=0 \\
z=h-\dfrac{1}{2}gt^2
\end{dcases}\]

Dans l'hémisphère Nord, on a \(\lambda>0\) donc \(\cos\lambda>0\) donc \(x>0\) : déviation vers l'Est.

De plus, comme \(z=h-\dfrac{1}{2}gt^2\), on a \(t=\sqrt{\dfrac{2\paren{h-z}}{g}}\).

D'où \[x\paren{z}=\dfrac{1}{3}\Omega\cos\paren{\lambda}g\paren{\dfrac{2\paren{h-z}}{g}}^{\nicefrac{3}{2}}.\]

On en déduit \[x_\maxi=x\paren{z=0}=\dfrac{1}{3}\Omega\cos\paren{\lambda}g\paren{\dfrac{2h}{g}}^{\nicefrac{3}{2}}.\]

Applications numériques dans les cas des expériences de Reich et Flammarion : \[x_\maxi=\SI{27.5}{\milli\metre}\qquad\text{et}\qquad x_\maxi=\SI{8}{\milli\metre}.\]

Ordre 2 :

On a \(\begin{dcases}
\dot{x}=\Omega\cos\paren{\lambda}gt^2 \\
\dot{y}=0 \\
\dot{z}=-gt
\end{dcases}\) donc \[\v{v}=\Omega\cos\paren{\lambda}gt^2\v{u}_x-gt\v{u}_z.\]

On en déduit \[\begin{aligned}
\v{F}_{iC}&=-2m\v{\Omega}\vecto\v{v} \\
&=\text{ }?
\end{aligned}\]

On obtient une nouvelle expression de \(\v{F}_{iC}\) plus précise et donc on peut calculer \(x,y,z\) à l'ordre 2.

On a appliqué la méthode des perturbations :

\begin{itemize}
    \item on calcule le mouvement non-perturbé (à l'ordre 0) en l'absence de \(\v{F}_{iC}\) ; \\
    \item on calcule \(\v{F}_{iC}\) à partir de l'ordre 0 et on perturbe celui-ci pour obtenir l'ordre 1 ; \\
    \item etc... \\
    \item on s'arrête lorsque le passage de l'ordre \(n\) à \(n+1\) produit un écart négligeable.
\end{itemize}

Autre effet du caractère non-galiléen du référentiel terrestre : le pendule de Foucault.


\chapter{Lois du frottement solide}

\minitoc

\note{À venir}


\part{Optique}

\chapter{Modèle scalaire des ondes lumineuses}

\minitoc

\note{À venir}


\chapter{Superposition de deux ondes lumineuses}

\minitoc

\note{À venir}


\chapter{Interféromètres par division du front d'onde}

\minitoc

\note{À venir}


\chapter{Interféromètres par division d'amplitude}

\minitoc

\note{À venir}


\part{Électromagnétisme}

\chapter*{Formulaire d'analyse vectorielle}
\addcontentsline{toc}{chapter}{Formulaire d'analyse vectorielle}

\minitoc

\section*{Calcul vectoriel}
\addcontentsline{toc}{section}{Calcul vectoriel}

On a les formules suivantes :

\begin{itemize}
    \item \(\rot\grad V=\v{0}\) \\
    \item \(\div\rot\v{A}=0\) \\
    \item \(\div\grad V=\lap V\) \\
    \item \(\rot\rot\v{A}=\grad\div\v{A}-\lapv\v{A}\) \\
    \item \(\grad\paren{V_1V_2}=V_1\grad V_2+V_2\grad V_1\) \\
    \item \(\rot\paren{V\v{A}}=V\rot\v{A}+\grad V\vecto\v{A}\) \\
    \item \(\div\paren{V\v{A}}=V\div\v{A}+\grad V\scal\v{A}\) \\
    \item \(\div\paren{\v{A}_1\vecto\v{A}_2}=\v{A}_2\scal\rot\v{A}_1-\v{A}_1\scal\rot\v{A}_2\)
\end{itemize}

\section*{Coordonnées cartésiennes}
\addcontentsline{toc}{section}{Coordonnées cartésiennes}

On a les formules suivantes :

\begin{itemize}
    \item \(\grad V=\pdv{V}{x}\v{u}_x+\pdv{V}{y}\v{u}_y+\pdv{V}{z}\v{u}_z\) \\
    \item \(\div\v{A}=\pdv{A_x}{x}+\pdv{A_y}{y}+\pdv{A_z}{z}\) \\
    \item \(\rot\v{A}=\paren{\pdv{A_z}{y}-\pdv{A_y}{z}}\v{u}_x+\paren{\pdv{A_x}{z}-\pdv{A_z}{x}}\v{u}_y+\paren{\pdv{A_y}{x}-\pdv{A_x}{y}}\v{u}_z\) \\
    \item \(\lap V=\pdv[order=2]{V}{x}+\pdv[order=2]{V}{y}+\pdv[order=2]{V}{z}\)
\end{itemize}

\section*{Coordonnées cylindriques}
\addcontentsline{toc}{section}{Coordonnées cylindriques}

On a les formules suivantes :

\begin{itemize}
    \item \(\grad V=\pdv{V}{r}\v{u}_r+\dfrac{1}{r}\pdv{V}{\theta}\v{u}_\theta+\pdv{V}{z}\v{u}_z\) \\
    \item \(\div\v{A}=\dfrac{1}{r}\pdv{rA_r}{r}+\dfrac{1}{r}\pdv{A_\theta}{\theta}+\pdv{A_z}{z}\) \\
    \item \(\rot\v{A}=\paren{\dfrac{1}{r}\pdv{A_z}{\theta}-\pdv{A_\theta}{z}}\v{u}_r+\paren{\pdv{A_r}{z}-\pdv{A_z}{r}}\v{u}_\theta+\dfrac{1}{r}\paren{\pdv{rA_\theta}{r}-\pdv{A_r}{\theta}}\v{u}_z\) \\
    \item \(\lap V=\dfrac{1}{r}\pdv{}{r}\paren{r\pdv{V}{r}}+\dfrac{1}{r^2}\pdv[order=2]{V}{\theta}+\pdv[order=2]{V}{z}\)
\end{itemize}

\section*{Coordonnées sphériques}
\addcontentsline{toc}{section}{Coordonnées sphériques}

\begin{itemize}
    \item \(\grad V=\pdv{V}{r}\v{u}_r+\dfrac{1}{r}\pdv{V}{\theta}\v{u}_\theta+\dfrac{1}{r\sin\theta}\pdv{V}{\phi}\v{u}_\phi\) \\
    \item \(\div\v{A}=\dfrac{1}{r^2}\pdv{r^2A_r}{r}+\dfrac{1}{r\sin\theta}\pdv{\sin\paren{\theta}A_\theta}{\theta}+\dfrac{1}{r\sin\theta}\pdv{A_\phi}{\phi}\) \\
    \item \(\rot\v{A}=\dfrac{1}{r\sin\theta}\paren{\pdv{\sin\paren{\theta}A_\phi}{\theta}-\pdv{A_\theta}{\phi}}\v{u}_r+\dfrac{1}{r}\paren{\dfrac{1}{\sin\theta}\pdv{A_r}{\phi}-\pdv{rA_\phi}{r}}\v{u}_\theta+\dfrac{1}{r}\paren{\pdv{rA_\theta}{r}-\pdv{A_r}{\theta}}\v{u}_\phi\) \\
    \item \(\lap V=\dfrac{1}{r}\pdv[order=2]{rV}{r}+\dfrac{1}{r^2\sin\theta}\pdv{}{\theta}\paren{\sin\paren{\theta}\pdv{V}{\theta}}+\dfrac{1}{r^2\sin^2\theta}\pdv[order=2]{V}{\phi}\)
\end{itemize}

\section*{Théorèmes}
\addcontentsline{toc}{section}{Théorèmes}

Théorème d'Ostrogradski-Green : \(S\) étant une surface fermée et \(\tau\) le volume intérieur à \(S\), on a \[\oiint_S\v{A}\scal\odif{\v{S}}=\iiint_\tau\paren{\div\v{A}}\odif{\tau}.\]

Théorème de Stokes-Ampère : \(C\) étant une courbe fermée bordant une surface \(S\), on a \[\oint_C\v{A}\scal\odif{\v{l}}=\iint_S\paren{\rot\v{A}}\scal\odif{\v{S}}.\]


\chapter{Électrostatique}

\minitoc

\note{À venir}


\chapter{Topographie du champ et du potentiel électrostatiques}

\minitoc

\note{À venir}


\chapter{Dipôle électrostatique}

\minitoc

\note{À venir}


\chapter{Magnétostatique}

\minitoc

\note{À venir}


\chapter{Dipôle magnétostatique}

\minitoc

\note{À venir}


\chapter{Équations de Maxwell}

\minitoc

\section*{Introduction}
\addcontentsline{toc}{section}{Introduction}

En électromagnétisme, on distingue :

\begin{itemize}
    \item le régime statique : \begin{itemize}
        \item électrostatique : théorème de Gauss (1831)
        \item magnétostatique : théorème d'Ampère (1820) \\
    \end{itemize}
    \item le régime variable : \begin{itemize}
        \item induction électromagnétique : loi de Lenz-Faraday (1831-1834).
    \end{itemize}
\end{itemize}

Gauss constate en 1831 l'absence de monopôle magnétique.

En 1865, James Clerk Maxwell réalise l'exploit de synthétiser les résultats expérimentaux en vingt équations scalaires à vingt inconnues écrites à l'aide de quaternions.

Ces équations sont reprises par Oliver Heaviside en 1884 pour aboutir aux quatre équations classiques (deux vectorielles, deux scalaires).

L'électromagnétisme (classique) devient une science aboutie jusqu'à l'arrivée des phénomènes quantiques qui amenèrent l'électrodynamique quantique (Feynman) et la théorie quantique des champs.

\section{Conservation de la charge}

\subsection{Première approche : 1D}

La charge \(q\) est un invariant : elle ne peut être ni créée ni annihilée.

\begin{center}
\begin{tikzpicture}
\draw[->] (0,0) -- (2,0) node[above,midway] {flux} node[below,midway] {entrant};
\draw (2,-1) -- (2,1) -- (5,1) node[below=10,midway] {volume \(V\)} -- (5,-1) -- (2,-1) node[above=10,midway] {chargé};
\draw[->] (5,0) -- (7,0) node[above,midway] {flux} node[below,midway] {sortant};
\end{tikzpicture}
\end{center}

La charge au temps \(t\) est égale à la charge initiale plus la charge entrante moins la charge sortante.

À une dimension, on a :

\begin{center}
\begin{tikzpicture}
\draw[->] (0,0) -- (10,0) node[below] {\(x\)};
\draw[->,blue] (0,0) -- (1,0) node[below] {\(\v{u}_x\)};
\draw (3,2) -- (7,2);
\draw (3,5) arc[start angle=90,end angle=270,x radius=1,y radius=1.5];
\draw[dotted] (4,3.5) arc[start angle=0,end angle=90,x radius=1,y radius=1.5];
\draw[dotted] (4,3.5) arc[start angle=0,end angle=-90,x radius=1,y radius=1.5];
\draw (8,3.5) arc[start angle=0,end angle=360,x radius=1,y radius=1.5];
\draw (3,5) -- (7,5);
\draw[dashed] (3,2) -- (3,0) node[below] {\(x\)};
\draw[dashed] (7,2) -- (7,0) node[below] {\(x+\odif{x}\)};
\draw (3,4.9) -- (5,5.5) node[above] {surface \(S\)};
\draw (7,4.9) -- (5,5.5);
\draw[->] (3,3.5) -- (4.5,3.5) node[above right] {\(\v{S}_\entrant\)};
\draw[->,red] (3,3.4) -- (4.5,3.4) node[below right] {\(\v{j}\paren{x,t}\)};
\draw[->] (7,3.5) -- (8.5,3.5) node[above right] {\(\v{S}_\sortant\)};
\draw[->,red] (7,3.4) -- (8.5,3.4) node[below right] {\(\v{j}\paren{x+\odif{x},t}\)};
\end{tikzpicture}
\end{center}

Le cylindre de surface \(S\) et de largeur \(\odif{x}\) est traversé par la densité de courant volumique \(\v{j}\).

\(\v{j}\paren{x,t}\) traverse la surface d'entrée du cylindre.

\(\v{j}\paren{x+\odif{x},t}\) traverse la surface de sortie du cylindre.

\(\rho\paren{x,t}\) : densité volumique de charge du cylindre.

Établissons le bilan des charges dans le cylindre entre \(t\) et \(t+\odif{t}\).

On note \(\fdif{Q_\entrant}\) et \(\fdif{Q_\sortant}\) la charge entrante (respectivement sortante) à travers la surface d'entrée (respectivement de sortie) du cylindre entre \(t\) et \(t+\odif{t}\).

On a \(\fdif{Q_\entrant}=I_\entrant\odif{t}\).

Or \(I_\entrant=\iint\v{j}\scal\odif{\v{S}}=j\paren{x,t}S\).

Donc \(\fdif{Q_\entrant}=j\paren{x,t}S\odif{t}\).

De même, on a \(\fdif{Q_\sortant}=j\paren{x+\odif{x},t}S\odif{t}\).

Donc \[\begin{aligned}
\fdif{Q}&=\fdif{Q_\entrant}-\fdif{Q_\sortant} \\
&=j\paren{x,t}S\odif{t}-j\paren{x+\odif{x},t}S\odif{t} \\
&=\underbrace{\paren{j\paren{x,t}-j\paren{x+\odif{x},t}}}_{-\pdv{j}{x}\odif{x}}S\odif{t}.
\end{aligned}\]

De plus, on a \(Q\paren{t}=\rho\paren{x,t}\odif{\tau}\) avec \(\odif{\tau}=S\odif{x}\) le volume du cylindre.

Pendant la durée \(\odif{t}\), la charge \(Q\paren{t}\) a varié de \[\begin{aligned}
\odif{Q}&=Q\paren{t+\odif{t}}-Q\paren{t} \\
&=\underbrace{\paren{\rho\paren{x,t+\odif{t}}-\rho\paren{x,t}}}_{\pdv{\rho}{t}\odif{t}}\odif{\tau}.
\end{aligned}\]

Or on a \[\begin{aligned}
\odif{Q}&=\fdif{Q} \\
\pdv{\rho}{t}\odif{t}S\odif{x}&=-\pdv{j}{x}\odif{x}S\odif{t} \\
\pdv{\rho}{t}&=-\pdv{j}{x} \\
\color{red}\pdv{\rho}{t}+\pdv{j}{x}&\color{red}=0.\color{black}
\end{aligned}\]

C'est l'équation locale de conservation de la charge à une dimension.

Remarque :

Ici, on a \(\v{j}\paren{x,t}=j\paren{x,t}\v{u}_x\).

On remarque \(\pdv{j}{x}=\div\v{j}\).

Donc l'équation locale de conservation de la charge se réécrit \[\color{red}\div\v{j}+\pdv{\rho}{t}=0.\color{black}\]

\subsection{Généralisation : 3D}

On admet que la formule précédente se généralise à trois dimensions : \[\color{red}\div\v{j}+\pdv{\rho}{t}=0.\color{black}\]

On peut le démontrer avec le théorème d'Ostrogradski-Green.

\section{Les équations de Maxwell}

\subsection{Énoncé}

Équation de Maxwell-Gauss \MG : \[\div\v{E}=\dfrac{\rho}{\epsilon_0}.\]

Équation de Maxwell-Flux ou Maxwell-Thomson \MT : \[\div\v{B}=0.\]

Équation de Maxwell-Faraday \MF : \[\rot\v{E}=-\pdv{\v{B}}{t}.\]

Équation de Maxwell-Ampère \MA : \[\rot\v{B}=\mu_0\v{j}+\mu_0\epsilon_0\pdv{\v{E}}{t}.\]

\subsection{Commentaires}

\begin{itemize}
    \item \MG et \MT sont des équations scalaires. \\
    \item \MF et \MA sont des équations vectorielles. \\
    \item \(\epsilon_0\) est la permittivité diélectrique du vide et \(\mu_0\) est la perméabilité magnétique du vide. On a \[\epsilon_0=\SI{8.85e-12}{\farad\per\metre}\qquad\text{et}\qquad\mu_0=\SI{4\pi e-7}{\henry\per\metre}.\] On verra que \(\mu_0\epsilon_0c^2=1\). \\
    \item \MG, \MT, \MF et \MA sont des équations linéaires : on peut appliquer le principe de superposition. \\
    \item \MG, \MT, \MF et \MA sont des équations locales, elles s'appliquent en un point \(M\), il n'y a pas d'expression intégrale. \\
    \item \MG et \MA relient \(\v{E}\) et \(\v{B}\) à leurs sources (charges et courants). \\
    \item \MF et \MA introduisent un couplage entre \(\v{E}\) et \(\v{B}\) en régime variable ; les champs \(\v{E}\) et \(\v{B}\) sont indissociables. \\
    \item \MG et \MF montrent que pour créer un champ \(\v{E}\), on peut utiliser des charges (\(\rho\)) ou un champ \(\v{B}\) variable (induction électromagnétique). \\
    \item \MA montre que pour créer un champ \(\v{B}\), on peut utiliser un courant (\(\v{j}\)) ou un champ électrique variable. \\
    \item On remarque que \(\v{j}\) est homogène à \(\epsilon_0\pdv{\v{E}}{t}=\v{j}_D\) : courant de déplacement.
\end{itemize}

\subsection{Compatibilité des équations de Maxwell}

On a \MA : \(\rot\v{B}=\mu_0\v{j}+\mu_0\epsilon_0\pdv{\v{E}}{t}\).

On applique \(\div\) : \[\begin{aligned}
\div\paren{\rot\v{B}}&=\div\paren{\mu_0\v{j}+\mu_0\epsilon_0\pdv{\v{E}}{t}} \\
&=0.
\end{aligned}\]

Donc \[\begin{WithArrows}
0&=\mu_0\paren{\div\v{j}+\epsilon_0\pdv{}{t}\div\v{E}} \Arrow{\MG} \\
&=\mu_0\paren{\div\v{j}+\pdv{\rho}{t}}.
\end{WithArrows}\]

Donc \(\div\v{j}+\pdv{\rho}{t}=0\).

Donc les équations de Maxwell sont compatibles avec l'équation locale de conservation de la charge.

\section{Forme intégrale des équations de Maxwell}

\subsection{Théorème de Gauss}

On considère une surface \(S\) fermée définissant le volume \(V\).

Soit \(\phi_{\v{E}}\) le flux de \(\v{E}\) à travers \(S\).

On a \(\phi_{\v{E}}=\oiint_S\v{E}\scal\odif{\v{S}}\).

D'après le théorème d'Ostrogradski-Green, on a \(\oiint_S\v{E}\scal\odif{\v{S}}=\iiint_V\div\v{E}\odif{\tau}\).

Donc \(\phi_{\v{E}}=\iiint_V\div\v{E}\odif{\tau}\).

D'après \MG, on a \(\div\v{E}=\dfrac{\rho}{\epsilon_0}\).

D'où \[\begin{aligned}
\phi_{\v{E}}&=\dfrac{1}{\epsilon_0}\iiint_V\rho\odif{\tau} \\
\color{red}\oiint_S\v{E}\scal\odif{\v{S}}&\color{red}=\dfrac{Q_\inte}{\epsilon_0}.\color{black}
\end{aligned}\]

On a prouvé le théorème de Gauss.

On dit que c'est la forme intégrale de \MG.

\subsection{Conservation du flux magnétique}

Une considère une surface \(S\) définissant le volume \(V\).

On a \[\begin{WithArrows}
\phi_{\v{B}}&=\oiint_S\v{B}\scal\odif{\v{S}} \Arrow{Ostrogradski-Green} \\
&=\iiint_V\div\v{B}\odif{\tau} \Arrow{\MT} \\
&=0.
\end{WithArrows}\]

D'où \[\color{red}\oiint_S\v{B}\scal\odif{\v{S}}=0.\color{black}\]

On a prouvé la conservation du flux magnétique.

On dit que c'est la forme intégrale de \MT.

\subsection{Loi de Lenz-Faraday}

Soit un contour fermé et orienté \(\Gamma\) s'appuyant sur la surface \(\v{S}\) dans une zone traversée par un champ \(\v{B}\).

On calcule \(e\) la circulation de \(\v{E}\) le long de \(\Gamma\) ; c'est la tension (ou force électromotrice) le long de \(\Gamma\). On a vu en MP2I la loi de Lenz-Faraday : \(e=-\odv{\phi}{t}\).

On a \[\begin{WithArrows}
e&=\oint_\Gamma\v{E}\scal\odif{\v{l}} \Arrow{Stokes-Ampère} \\
&=\iint_S\rot\v{E}\scal\odif{\v{S}} \Arrow{\MF} \\
&=\iint_S\paren{-\pdv{\v{B}}{t}}\scal\odif{\v{S}} \\
&=-\pdv{}{t}\iint_S\v{B}\scal\odif{\v{S}} \Arrow{\(\phi=\iint_S\v{B}\scal\odif{\v{S}}\) : flux magnétique} \\
&=-\pdv{\phi}{t} \Arrow{\(\phi\) ne dépend que du temps} \\
&=-\odv{\phi}{t}.
\end{WithArrows}\]

On a prouvé la loi de Lenz-Faraday.

Remarque : comme dans la loi de Lenz-Faraday, dans \MF : \(\rot\v{E}=-\pdv{\v{B}}{t}\), le signe \guillemets{\(-\)} est dû au principe de modération.

\subsection{Théorème d'Ampère généralisé}

Soit un contour fermé et orienté \(\Gamma\) s'appuyant sur la surface \(\v{S}\) dans une zone traversée par un champ \(\v{B}\).

On a \[\begin{WithArrows}
\call{C}&=\oint_\Gamma\v{B}\scal\odif{\v{l}} \Arrow{Stokes-Ampère} \\
&=\iint_S\rot\v{B}\scal\odif{\v{S}} \Arrow{\MA} \\
&=\iint_S\mu_0\v{j}\scal\odif{\v{S}}+\iint_S\mu_0\epsilon_0\pdv{\v{E}}{t}\scal\odif{\v{S}} \\
&=\mu_0\iint_S\v{j}\scal\odif{\v{S}}+\mu_0\epsilon_0\pdv{}{t}\iint_S\v{E}\scal\odif{\v{S}} \\
&\color{red}=\mu_0I_\enl+\mu_0\epsilon_0\pdv{}{t}\iint_S\v{E}\scal\odif{\v{S}}.
\end{WithArrows}\]

C'est le théorème d'Ampère généralisé.

Remarque : en statique, on a \(\pdv{}{t}=0\) donc on retrouve le théorème d'Ampère \(\call{C}=\mu_0I_\enl\).

\subsection{Bilan}

\begin{center}
\begin{Tabular}{c|c}
\thead{Loi locale} & \thead{Loi intégrale} \\
\hline \\
\MG : \(\div\v{E}=\dfrac{\rho}{\epsilon_0}\) & Théorème de Gauss : \(\oiint\v{E}\scal\odif{\v{S}}=\dfrac{Q_\inte}{\epsilon_0}\) \\\\
\MT : \(\div\v{B}=0\) & Conservation du flux magnétique : \(\oiint\v{B}\scal\odif{\v{S}}=0\) \\\\
\MF : \(\rot\v{E}=-\pdv{\v{B}}{t}\) & Loi de Lenz-Faraday : \(e=\oint\v{E}\scal\odif{\v{l}}=-\odv{\phi}{t}\) \\\\
\MA : \(\rot\v{B}=\mu_0\v{j}+\mu_0\epsilon_0\pdv{\v{E}}{t}\) & \makecell{Théorème d'Ampère généralisé : \\ \(\call{C}=\oint\v{B}\scal\odif{\v{l}}=\mu_0I_\enl+\mu_0\epsilon_0\pdv{}{t}\iint\v{E}\scal\odif{\v{S}}\)}
\end{Tabular}
\end{center}

\note{Chapitre incomplet : progression du 20/11}


\chapter{Énergie du champ électromagnétique}

\minitoc

\note{À venir}

\end{document}