\chapter{Énergie du champ électromagnétique}

\minitoc

\section*{Introduction}
\addcontentsline{toc}{section}{Introduction}

Ce chapitre traite des échanges d'énergie entre la matière et le champ électromagnétique.

\section{Première approche : le conducteur ohmique}

\subsection{Modèle de Drude, loi d'Ohm locale}

On cherche à construire un modèle microscopique de la conduction électrique dans un conducteur électrique. Le modèle suivant a été proposé en 1900 par Paul Drude (Allemagne) et amélioré en 1905 par Hendrik Lorentz (Pays-Bas, Prix Nobel de Physique en 1902).

Le modèle de Drude-Lorentz explique microscopiquement la loi empirique d'Ohm (1827).

Soit un conducteur électrique immobile possédant des charges mobiles (souvent des électrons) de charge \(q\), de masse \(m\) et de densité volumique \(n\).

Sous l'effet d'un champ \(\v{E}\) extérieur, les porteurs de charge se mettent en mouvement à la vitesse \(\v{v}\).

\begin{center}
\begin{tikzpicture}[scale=0.8]
\draw (3,2) -- (7,2);
\draw (3,5) arc[start angle=90,end angle=270,x radius=1,y radius=1.5];
\draw[dotted] (4,3.5) arc[start angle=0,end angle=90,x radius=1,y radius=1.5];
\draw[dotted] (4,3.5) arc[start angle=0,end angle=-90,x radius=1,y radius=1.5];
\draw (8,3.5) arc[start angle=0,end angle=360,x radius=1,y radius=1.5];
\draw (3,5) -- (7,5);
\draw[->,blue] (4,5.5) -- ++(2,0) node[above] {\(\v{E}\)};
\draw[->,blue] (6,1.5) -- ++(2,0) node[above] {\(\v{E}\)};
\draw[->,green] (4.1,3.5) -- ++(1.5,0) node[below] {\(\v{v}\)};
\end{tikzpicture}
\end{center}

On modélise l'interaction entre les porteurs de charge et le milieu par une force de frottement \(\v{f}=-\lambda\v{v}\).

On applique le \PFD sur un porteur de charge : \[m\odv{\v{v}}{t}=q\v{E}-\lambda\v{v}\] où \(q\v{E}\) est la composante électrique de la force de Lorentz.

On a donc \(\odv{\v{v}}{t}+\dfrac{\lambda}{m}\v{v}=\dfrac{q}{m}\v{E}\).

On pose \(\tau=\dfrac{m}{\lambda}\) et on a \[\v{v}\paren{t}=\v{k}\e{-\nicefrac{t}{\tau}}+\dfrac{q}{\lambda}\v{E}.\]

On détermine \(\v{k}\) avec les conditions initiales.

Si \(t\geq5\tau\), on a \(\v{v}\paren{t}=\dfrac{q}{\lambda}\v{E}=\v{v}_\limite\) : le régime transitoire disparaît et on atteint le régime permanent.

Dans un conducteur classique (\ie un métal), on a \(\tau\approx\SI{e-14}{\second}\) donc on peut légitimement considérer que \(\v{v}=\v{v}_\limite\).

On note \(\v{j}\) la densité de courant volumique dans le conducteur. On a \[\v{j}=nq\v{v}.\]

Ici, \(\v{j}=nq\v{v}_\limite=\dfrac{nq^2}{\lambda}\v{E}\).

C'est la loi d'Ohm locale : \[\color{red}\v{j}=\dfrac{nq^2}{\lambda}\v{E}.\]

On pose \(\gamma=\dfrac{nq^2}{\lambda}\) la conductivité du milieu (en \(\unit{\siemens\per\metre}\)) et on a \[\color{red}\v{j}=\gamma\v{E}.\]

On utilise aussi parfois \(\rho=\dfrac{1}{\gamma}\) la résistivité du milieu (en \(\unit{\ohm\metre}\)).

Voici quelques ordres de grandeur de conductivité :

\begin{center}
\begin{Tabular}{|c|c|c|}
\hline
\thead{\textbf{Milieu}} & \thead{\textbf{\(\gamma\) (\(\unit{\siemens\per\metre}\))}} & \thead{\textbf{Nature}} \\
Verre & \(\num{e-11}\) & Isolant \\
Paraffine & \(\num{e-8}\) & Isolant \\
Eau potable & \(\numrange{e-4}{e-2}\) & Mauvais conducteur \\
Fer & \(\num{9.9e6}\) & Bon conducteur \\
\textbf{Cuivre} & \textbf{\(\num{5.8e7}\)} & \textbf{Excellent conducteur} \\
Argent & \(\num{6.2e7}\) & Excellent conducteur \\
Supraconducteur & \(\pinf\) & Conducteur parfait \\
\hline
\end{Tabular}
\end{center}

Les matériaux dont la conductivité est inférieure à \(\num{e-4}\) sont appelés des isolants.

Ceux dont la conductivité est comprise entre \(\num{e-4}\) et \(\num{e4}\) sont appelés des semi-conducteurs.

Ceux dont la conductivité est supérieure à \(\num{e4}\) sont appelés des conducteurs.

On avait \(\tau=\dfrac{m}{\lambda}\) donc \(\gamma=\dfrac{nq^2}{\lambda}=\dfrac{nq^2\tau}{m}\) donc \(\tau=\dfrac{\gamma m}{nq^2}\).

Considérons l'exemple du cuivre. Dans le cuivre, les porteurs de charge sont les électrons donc on a \(\gamma=\SI{5.8e7}{\siemens\per\metre}\), \(m=\SI{9.1e-31}{\kilo\gram}\), \(q=-e=\SI{-1.6e-19}{\coulomb}\) et \(n=\SI{10e29}{\per\cubic\metre}\) et on en déduit \[\tau\approx\dfrac{\num{e-30}\times\num{e6}}{\num{e29}\times\num{e-38}}=\SI{e-15}{\second}.\]

\subsection{Loi d'Ohm intégrale}

Soit une portion de conducteur ohmique de longueur \(L\) et de section \(S\) placée dans un champ électrique extérieur stationnaire \(\v{E}=E\v{u}_z\) créé par un générateur extérieur.

\begin{center}
\begin{tikzpicture}[scale=0.8]
\draw[dashed] (0,3.5) -- (12,3.5);
\draw (3,2) -- (9,2);
\draw (3,5) arc[start angle=90,end angle=270,x radius=1,y radius=1.5];
\draw[dotted] (4,3.5) arc[start angle=0,end angle=90,x radius=1,y radius=1.5];
\draw[dotted] (4,3.5) arc[start angle=0,end angle=-90,x radius=1,y radius=1.5];
\draw[pattern=north east lines, pattern color=gray] (10,3.5) arc[start angle=0,end angle=360,x radius=1,y radius=1.5];
\draw (3,5) -- (9,5);
\draw[<->] (3,1) -- (9,1) node[midway,below] {\(L\)};
\draw[->,red] (5,3.5) -- ++(1.5,0) node[below] {\(\v{j}\)};
\filldraw (3,3.5) circle (2pt) node[above] {\(A\)};
\filldraw (9,3.5) circle (2pt) node[above] {\(B\)};
\node at (9,2.5) {\(S\)};
\draw[->,blue] (10.8,3.5) -- (11.8,3.5) node[below] {\(\v{u}_z\)};
\end{tikzpicture}
\end{center}

On a \(\v{j}=\gamma\v{E}\).

On pose \(I\) l'intensité traversant une surface d'entrée (ou sortie) du conducteur : \[I=\iint_{\begin{tikzpicture}[scale=0.13]\draw[pattern=north east lines, pattern color=gray] (10,3.5) arc[start angle=0,end angle=360,x radius=1,y radius=1.5];\end{tikzpicture}}\v{j}\scal\odif{\v{S}}=\iint_{\begin{tikzpicture}[scale=0.13]\draw[pattern=north east lines, pattern color=gray] (10,3.5) arc[start angle=0,end angle=360,x radius=1,y radius=1.5];\end{tikzpicture}}\gamma E\odif{S}=\gamma ES.\]

On pose \(U\) la différence de potentiel entre les surfaces d'entrée et de sortie du conducteur : \[U=V_A-V_B.\]

On a \[\begin{aligned}
\v{E}&=-\grad V \\
E\v{u}_z&=-\odv{V}{z}\v{u}_z \\
-E\odif{z}&=\odif{V}
\end{aligned}\] donc \[\int_A^B\odif{V}=\int_A^B-E\odif{z}\] donc \[V\paren{B}-V\paren{A}=-U=-EL\] \ie \[U=EL.\]

Ainsi, on a \[I=\gamma\dfrac{U}{L}S\] et donc \[U=\dfrac{L}{\gamma S}I=RI\] en posant \(R=\dfrac{L}{\gamma S}\) la résistance du conducteur de longueur \(L\) et de section \(S\) (en \(\unit{\ohm}\)).

D'où la loi d'Ohm : \[\color{red}U=RI.\]

\subsection{Puissance cédée par le champ électromagnétique aux porteurs de charge}

\subsubsection{Cas d'une particule ponctuelle}

Soit une particule ponctuelle animée de la vitesse \(\v{v}\) portant la charge \(q\) et placée dans le champ électromagnétique \(\paren{\v{E},\v{B}}\).

En notant \(P\) la puissance de la force de Lorentz, on a \[\begin{aligned}
P&=\v{F}\scal\v{v} \\
&=q\paren{\v{E}+\v{v}\vecto\v{B}}\scal\v{v} \\
&=q\v{E}\scal\v{v}+q\underbrace{\v{v}\vecto\v{B}\scal\v{v}}_{=0} \\
\color{red}P&\color{red}=q\v{E}\scal\v{v}.
\end{aligned}\]

\textcolor{red}{La puissance de la force magnétique est nulle.}

\(P\) est la puissance cédée par le champ électromagnétique au porteur de charge.

\subsubsection{Expression volumique}

On raisonne sur un élément de volume \(\odif{\tau}\) qui contient des particules chargées de charge \(q\), animées de la vitesse \(\v{v}\) et de densité volumique \(n\).

En notant \(\odif{P}\) la puissance élémentaire cédée par le champ aux porteurs de charge dans le volume \(\odif{\tau}\), on a \[\odif{P}=\underbrace{n\odif{\tau}}_{\substack{\text{nombre} \\ \text{de} \\ \text{porteurs}}}\underbrace{q\v{E}\scal\v{v}.}_{\substack{\text{puissance} \\ \text{cédée à} \\ \text{un porteur}}}\]

On a alors \[\begin{aligned}
\odv{P}{\tau}&=nq\v{E}\scal\v{v} \\
&=nq\v{v}\scal\v{E} \\
\color{red}\odv{P}{\tau}&\color{red}=\v{j}\scal\v{E}.
\end{aligned}\]

Cette grandeur est une puissance volumique (en \(\unit{\watt\per\cubic\metre}\)).

Remarque : on a raisonné avec un seul type de porteur de charge. S'il y en a plusieurs, on a \[\odif{P}=\sum_i\odif{P_i}=\sum_in_i\odif{\tau}q_i\v{E}\scal\v{v}_i\] et \[\odv{P}{\tau}=\paren{\sum_i\v{j}_i}\scal\v{E}=\v{j}\scal\v{E}.\]

\subsubsection{Application : effet Joule}

On a \(\odv{P}{\tau}=\v{j}\scal\v{E}\).

Avec un conducteur ohmique, d'après la loi d'Ohm, on a \(\v{j}=\gamma\v{E}\).

Ainsi, on a la loi de Joule (locale) : \[\color{red}\odv{P}{\tau}=\gamma E^2.\]

Considérons l'exemple ci-dessous :

\begin{center}
\begin{tikzpicture}[scale=0.8]
\draw (3,2) -- (9,2);
\draw (3,5) arc[start angle=90,end angle=270,x radius=1,y radius=1.5];
\draw[dotted] (4,3.5) arc[start angle=0,end angle=90,x radius=1,y radius=1.5];
\draw[dotted] (4,3.5) arc[start angle=0,end angle=-90,x radius=1,y radius=1.5];
\draw (10,3.5) arc[start angle=0,end angle=360,x radius=1,y radius=1.5];
\draw (3,5) -- (9,5);
\draw[<->] (3,1) -- (9,1) node[midway,below] {\(L\)};
\draw[->,red] (4.7,3) -- ++(1.5,0) node[below] {\(\v{j}\)};
\draw[->,blue] (5,4) -- ++(2.1,0) node[above] {\(\v{E}\)};
\node at (9,2.5) {\(S\)};
\end{tikzpicture}
\end{center}

On cherche \(P_\tot\) la puissance totale cédée par le champ aux porteurs de charge dans le conducteur ohmique de longueur \(L\) et de section \(S\).

On a \(\odv{P}{\tau}=\gamma E^2\) donc \(\odif{P}=\gamma E^2\odif{\tau}\). Ainsi \[P_\tot=\iiint_V\odif{P}=\gamma E^2\iiint_V\odif{\tau}=\gamma E^2V.\]

Or \(V=SL\) donc \[P_\tot=\gamma E^2SL=\dfrac{\gamma E^2}{L}SL^2.\]

Or \(U=EL\), \(R=\dfrac{L}{\gamma S}\) et \(U=RI\) donc \[P_\tot=\dfrac{1}{R}U^2=RI^2.\]

Ainsi, la puissance dissipée par effet Joule est égale à la puissance cédée par le champ aux porteurs de charge.

\section{Bilan d'énergie électromagnétique}

\subsection{Équation locale de conservation}

On considère une surface \(S\) définissant le volume \(V\).

On note \(U\paren{t}\) l'énergie électromagnétique emmagasinée dans le volume \(V\), \(\fdif{W_\sortie}\) l'énergie qui a traversé la surface \(S\) de l'intérieur vers l'extérieur entre \(t\) et \(t+\odif{t}\) et \(\fdif{W_\cedee}\) l'énergie cédée par le champ aux porteurs de charge dans le volume \(V\) entre \(t\) et \(t+\odif{t}\).

On a \[\color{red}\odif{U}=U\paren{t+\odif{t}}-U\paren{t}=-\fdif{W_\sortie}-\fdif{W_\cedee}.\]

On définit \(u\paren{t}=\odv{U}{\tau}\) la densité d'énergie électromagnétique (en \(\unit{\joule\per\cubic\metre}\)).

On a \[\begin{aligned}
\odif{U}&=u\odif{\tau} \\
U&=\iiint_Vu\odif{\tau}
\end{aligned}\] et \[\fdif{W_\cedee}=\iiint_V\v{j}\scal\v{E}\odif{\tau}\odif{t}.\]

On définit \(\v{\Pi}\) la densité de puissance électromagnétique surfacique (en \(\unit{\watt\per\square\metre}\)) et on a \[\fdif{W_\sortie}=\oiint_S\v{\Pi}\scal\odif{\v{S}}\odif{t}.\]

De plus, on a \[\odif{U}=\pdv{}{t}\paren{\iiint_Vu\odif{\tau}}\odif{t}=\iiint_V\pdv{u}{t}\odif{\tau}\odif{t}.\]

Donc \[\begin{WithArrows}
\iiint_V\pdv{u}{t}\odif{\tau}\odif{t}&=-\iiint_V\v{j}\scal\v{E}\odif{\tau}\odif{t}-\oiint_S\v{\Pi}\scal\odif{\v{S}} \Arrow{Ostrogradski-Green} \\
\iiint_V\pdv{u}{t}\odif{\tau}\odif{t}&=-\iiint_V\v{j}\scal\v{E}\odif{\tau}\odif{t}-\iiint_V\div\v{\Pi}\odif{\tau}\odif{t} \\
\iiint_V\paren{\pdv{u}{t}+\v{j}\scal\v{E}+\div\v{\Pi}}\odif{\tau}\odif{t}&=0
\end{WithArrows}\] soit \textcolor{red}{l'équation locale de conservation de la charge} : \[\color{red}\pdv{u}{t}+\v{j}\scal\v{E}+\div\v{\Pi}=0.\]

On admet temporairement \[u\paren{t}=\dfrac{1}{2}\epsilon_0E^2+\dfrac{B^2}{2\mu_0}\] et \[\v{\Pi}=\dfrac{\v{E}\vecto\v{B}}{\mu_0}.\]

On appelle \(\v{\Pi}\) le vecteur de Poynting.

\subsection{Identité de Poynting (HP)}

D'après \MA, on a \[\begin{aligned}
\v{j}&=\dfrac{1}{\mu_0}\paren{\rot\v{B}-\mu_0\epsilon_0\pdv{\v{E}}{t}} \\
\v{j}\scal\v{E}&=\dfrac{\v{E}}{\mu_0}\paren{\rot\v{B}-\mu_0\epsilon_0\pdv{\v{E}}{t}}.
\end{aligned}\]

Or \(\div\paren{\v{E}\vecto\v{B}}=\v{B}\scal\rot\v{E}-\v{E}\scal\rot\v{B}\) donc \[\begin{WithArrows}
\v{j}\scal\v{E}&=\dfrac{1}{\mu_0}\paren{\v{B}\scal\rot\v{E}-\div\paren{\v{E}\vecto\v{B}}-\mu_0\epsilon_0\v{E}\scal\pdv{\v{E}}{t}} \Arrow{\MF} \\
&=\dfrac{1}{\mu_0}\biggl(-\div\paren{\v{E}\vecto\v{B}}-\underbrace{\v{B}\scal\pdv{\v{B}}{t}}_{=\pdv{}{t}\paren{\dfrac{B^2}{2}}}-\mu_0\epsilon_0\underbrace{\v{E}\scal\pdv{\v{E}}{t}}_{=\pdv{}{t}\paren{\dfrac{E^2}{2}}}\biggr).
\end{WithArrows}\]

On retrouve l'\textcolor{red}{équation locale de conservation de la charge} ou \textcolor{red}{identité de Poynting} : \[\color{red}\v{j}\scal\v{E}+\div\dfrac{\v{E}\vecto\v{B}}{\mu_0}+\pdv{}{t}\paren{\dfrac{B^2}{2\mu_0}+\dfrac{\epsilon_0E^2}{2}}=0.\]

Remarque : le vecteur de Poynting \(\v{\Pi}\) est parfois noté \(\v{R}\).

\subsection{Ordres de grandeur}

On considère une surface \(\odif{\v{S}}\) traversée par le vecteur de Poynting \(\v{\Pi}\) et on note \(\theta\) l'angle entre \(\odif{\v{S}}\) et \(\v{\Pi}\).

En notant \(\odif{P_R}\) la puissance rayonnée par le vecteur de Poynting à travers la surface \(\odif{S}\), on a \[\odif{P_R}=\v{\Pi}\scal\odif{\v{S}}=\Pi\odif{S}\cos\theta.\]

On en déduit \[P_R=\iint\Pi\odif{S}\cos\theta\] et, si la surface est plane et \(\v{\Pi}\) est uniforme : \[P_R=\Pi S\cos\theta.\]

Calculons par exemple la puissance surfacique rayonnée par le Soleil en haut de l'atmosphère terrestre.

On note \(D_{TS}\) la distance entre le Soleil et la Terre et \(R_S\) le rayon du Soleil.

On note \(T_S=\SI{5800}{\kelvin}\) la température du Soleil.

En notant \(\v{\Pi}_S\) la puissance surfacique rayonnée par le Soleil, d'après la loi de Stefan, on a \[\Pi_S=\sigma T_S^4\] avec \(\sigma=\SI{5.67e-8}{\watt\per\metre\squared\per\kelvin\tothe{4}}\), \ie \(\Pi_S\approx\SI{50}{\mega\watt\per\metre\squared}\).

On note \(\Pi_T\) le flux solaire au niveau de la Terre et \(P_S\) la puissance totale émise par le Soleil. On a \[\begin{aligned}
P_S&=\oiint_{\substack{\text{surface} \\ \text{Soleil}}}\v{\Pi}_S\scal\odif{\v{S}}=4\pi R_S^2\Pi_S \\
&=\oiint_{\substack{\text{sphère} \\ \text{de} \\ \text{rayon }D_{TS}}}\v{\Pi}_T\scal\odif{\v{S}}=4\pi D_{TS}^2\Pi_T.
\end{aligned}\]

Donc \(4\pi R_S^2\Pi_S=4\pi D_{TS}^2\Pi_T\).

Donc \(\Pi_T=\Pi_S\paren{\dfrac{R_S}{D_{TS}}}^2\).

Or \(D_{TS}=\SI{150e6}{\kilo\metre}\) et \(R_S=\SI{695e3}{\kilo\metre}\) donc \[\color{red}\Pi_T=\SI{1368}{\watt\per\metre\squared}.\]

\section{Exemples de bilan d'énergie électromagnétique}

\subsection{Conducteur ohmique}

\begin{center}
\begin{tikzpicture}[scale=0.8]
\draw (3,2) -- (9,2);
\draw (3,5) arc[start angle=90,end angle=270,x radius=1,y radius=1.5];
\draw[dotted] (4,3.5) arc[start angle=0,end angle=90,x radius=1,y radius=1.5];
\draw[dotted] (4,3.5) arc[start angle=0,end angle=-90,x radius=1,y radius=1.5];
\draw (10,3.5) arc[start angle=0,end angle=360,x radius=1,y radius=1.5];
\draw (3,5) -- (9,5);
\draw[<->] (3,1) -- (9,1) node[midway,below] {\(L\)};
\draw[->,red] (4.7,3) -- ++(1.5,0) node[below] {\(\v{j}\)};
\draw[->,blue] (5,4) -- ++(2.1,0) node[above] {\(\v{E}\)};
\node at (9,2.5) {\(S\)};
\draw (3,3.5) -- (3,5) node[midway,right] {\(a\)};
\filldraw (11,6.5) circle (2pt) node[below] {\(M\)};
\draw[->,blue] (11,6.5) -- ++(1,0) node[below] {\(\v{u}_z\)};
\draw[->,blue] (11,6.5) -- ++(0,1) node[left] {\(\v{u}_r\)};
\filldraw[blue] (10.4,6.5) circle (2pt);
\draw[blue] (10.4,6.5) circle (6pt) node[left=6pt] {\(\v{u}_\theta\)};
\filldraw[green] (6,6.2) circle (2pt);
\draw[green] (6,6.2) circle (6pt) node[left=6pt] {\(\v{B}\)};
\draw[red,->] (4.3,1.7) -- ++(0,0.9);
\draw[red,->] (5.3,1.7) -- ++(0,0.9);
\draw[red,->] (7,1.7) -- ++(0,0.9);
\draw[red,->] (8,1.7) -- ++(0,0.9);
\draw[red,->] (4.3,5.3) node[above left] {\(\v{\Pi}\)} -- ++(0,-0.9);
\draw[red,->] (5.3,5.3) -- ++(0,-0.9);
\draw[red,->] (6.3,5.3) -- ++(0,-0.9);
\draw[red,->] (7.6,5.3) -- ++(0,-0.9);
\end{tikzpicture}
\end{center}

\(\v{E}\) est un champ électrique extérieur.

On a \(\v{j}=\gamma\v{E}\) et \(U=RI\) où \(R=\dfrac{L}{\gamma S}\).

On a \(I=jS=j\pi a^2=\gamma E\pi a^2\).

On a donc \(\v{B}=\begin{dcases}
\dfrac{\mu_0I}{2\pi a^2}r\v{u}_\theta=\dfrac{\mu_0\gamma Er}{2}\v{u}_\theta &\text{si }r<a \\
\dfrac{\mu_0I}{2\pi r}\v{u}_\theta=\dfrac{\mu_0\gamma Ea^2}{2}\v{u}_\theta &\text{si }r>a
\end{dcases}\).

Alors, \(\v{\Pi}=\dfrac{\v{E}\vecto\v{B}}{\mu_0}=\begin{dcases}
\dfrac{E}{\mu_0}\v{u}_z\vecto\dfrac{\mu_0\gamma Er}{2}\v{u}_\theta=-\dfrac{\gamma E^2r}{2}\v{u}_r &\text{si }r<a \\
\dfrac{E}{\mu_0}\v{u}_z\vecto\dfrac{\mu_0\gamma Ea^2}{2r}\v{u}_\theta=-\dfrac{\gamma E^2a^2}{2r}\v{u}_r &\text{si }r>a
\end{dcases}\)

Ainsi, \(\v{\Pi}\) est dirigé selon \(-\v{u}_r\).

\begin{itemize}
    \item Contrairement à l'intuition, la puissance du champ \(\paren{\v{E},\v{B}}\) entre par la surface latérale. \\
    \item Le conducteur ohmique est dissipatif donc la puissance doit y entrer, d'où \(\v{\Pi}\) selon \(-\v{u}_r\).
\end{itemize}

En notant \(P\) la puissance entrant dans le conducteur ohmique, on a \[\begin{aligned}
P&=\oiint\v{\Pi}\scal\odif{\v{S}} \\
&=\cancelto{0}{\iint_{S_g}\v{\Pi}\scal\odif{\v{S}}}+\cancelto{0}{\iint_{S_d}\v{\Pi}\scal\odif{\v{S}}}+\iint_{S_l}\v{\Pi}\scal\odif{\v{S}} \\
&=\iint_{S_l}\Pi\odif{S} \\
&=-\dfrac{\gamma E^2a^2}{2a}2\pi aL \\
&=-\dfrac{\gamma E^2a^2\pi L^2}{L} \\
&=-\dfrac{\gamma U^2\pi a^2}{L} \\
&=-\dfrac{U^2}{R}.
\end{aligned}\]

Bilan de puissance : \[\iiint\pdv{u}{t}\odif{\tau}\odif{t}+\underbrace{\iiint\v{j}\scal\v{E}\odif{\tau}\odif{t}}_{=RI^2}+\underbrace{\oiint\v{\Pi}\scal\odif{\v{S}}\odif{t}}_{=-RI^2}=0.\]

Or \(u=\dfrac{B^2}{2\mu_0}+\dfrac{\epsilon_0E^2}{2}\) donc \(\pdv{u}{t}=0\).

D'où \(0=0\) : le bilan de puissance est valide.
